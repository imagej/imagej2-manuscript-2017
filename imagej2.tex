%% BioMed_Central_Tex_Template_v1.06

%%IMPORTANT: do not delete the first line of this template
%%It must be present to enable the BMC Submission system to
%%recognise this template!!

%%%%%%%%%%%%%%%%%%%%%%%%%%%%%%%%%%%%%%%%%
%%                                     %%
%%  LaTeX template for BioMed Central  %%
%%     journal article submissions     %%
%%                                     %%
%%          <8 June 2012>              %%
%%                                     %%
%%                                     %%
%%%%%%%%%%%%%%%%%%%%%%%%%%%%%%%%%%%%%%%%%


%%%%%%%%%%%%%%%%%%%%%%%%%%%%%%%%%%%%%%%%%%%%%%%%%%%%%%%%%%%%%%%%%%%%%
%%                                                                 %%
%% For instructions on how to fill out this Tex template           %%
%% document please refer to Readme.html and the instructions for   %%
%% authors page on the biomed central website                      %%
%% http://www.biomedcentral.com/info/authors/                      %%
%%                                                                 %%
%% Please do not use \input{...} to include other tex files.       %%
%% Submit your LaTeX manuscript as one .tex document.              %%
%%                                                                 %%
%% All additional figures and files should be attached             %%
%% separately and not embedded in the \TeX\ document itself.       %%
%%                                                                 %%
%% BioMed Central currently use the MikTex distribution of         %%
%% TeX for Windows) of TeX and LaTeX.  This is available from      %%
%% http://www.miktex.org                                           %%
%%                                                                 %%
%%%%%%%%%%%%%%%%%%%%%%%%%%%%%%%%%%%%%%%%%%%%%%%%%%%%%%%%%%%%%%%%%%%%%

%%% additional documentclass options:
%  [doublespacing]
%  [linenumbers]   - put the line numbers on margins

%%% loading packages, author definitions

%\documentclass[twocolumn]{bmcart}% uncomment this for twocolumn layout and comment line below
\documentclass{bmcart}

%%% Load packages
%\usepackage{amsthm,amsmath}
%\RequirePackage{natbib}
%\RequirePackage[authoryear]{natbib}% uncomment this for author-year bibliography
%\RequirePackage{hyperref}
\usepackage[utf8]{inputenc} %unicode support
%\usepackage[applemac]{inputenc} %applemac support if unicode package fails
%\usepackage[latin1]{inputenc} %UNIX support if unicode package fails

%% We use the glossaries package for the list of acronyms.
\usepackage[nonumberlist]{glossaries}
\renewcommand{\glossarypreamble}{\footnotesize}
\makeglossaries
\newacronym{alida}{Alida}{Advanced Library for Integrated Development of data analysis Applications}
\newacronym{api}{API}{Application Programming Interface}
\newacronym{awt}{AWT}{Abstract Windowing Toolkit}
\newacronym{bar}{BAR}{Broadly Applicable Routines}
\newacronym{bmc}{BMC}{BioMed Central}
\newacronym{bsd}{BSD}{Berkeley Software Distribution}
\newacronym{ci}{CI}{continuous integration}
\newacronym{cpu}{CPU}{Central Processing Unit}
\newacronym{di}{DI}{dependency injection}
\newacronym{dry}{DRY}{Don't Repeat Yourself}
\newacronym{epfl}{EPFL}{École Polytechnique Fédérale de Lausanne}
\newacronym{fiji}{Fiji}{Fiji Is Just ImageJ}
\newacronym{gimp}{GIMP}{GNU Image Manipulation Program}
\newacronym{gnu}{GNU}{GNU's Not Unix}
\newacronym{gpu}{GPU}{Graphics Processing Unit}
\newacronym{gtk}{GTK}{GIMP ToolKit}
\newacronym{http}{HTTP}{Hypertext Transfer Protocol}
\newacronym{io}{I/O}{Input/Output}
\newacronym{ide}{IDE}{Integrated Development Environment}
\newacronym{ioc}{IoC}{inversion of control}
\newacronym{itk}{ITK}{Insight ToolKit}
\newacronym{jar}{JAR}{Java ARchive}
\newacronym{jit}{JIT}{Just-In-Time compiler}
\newacronym{jvm}{JVM}{Java Virtual Machine}
\newacronym{knime}{KNIME}{KoNstanz Information MinEr}
\newacronym{knip}{KNIP}{KNIME Image Processing}
\newacronym{laf}{L\&F}{Look \& Feel}
\newacronym{list}{LIST}{Luxembourg Institute of Science and Technology}
\newacronym{lut}{LUT}{color lookup table}
\newacronym{loci}{LOCI}{Laboratory for Optical and Computational Instrumentation}
\newacronym{macos}{macOS}{Macintosh Operating System}
\newacronym{matlab}{MATLAB}{MATrix LABoratory}
\newacronym{mitobo}{MiToBo}{Microscopy image analysis ToolBox}
\newacronym{nih}{NIH}{National Institutes of Health}
\newacronym{ome}{OME}{Open Microscopy Environment}
\newacronym{omero}{OMERO}{OME Remote Objects}
\newacronym{opencv}{OpenCV}{Open source Computer Vision library}
\newacronym{pom}{POM}{Project Object Model}
\newacronym{ram}{RAM}{Random-Access Memory}
\newacronym{rest}{RESTful}{REpresentational State Transfer}
\newacronym[longplural={regions of interest}]{roi}{ROI}{region of interest}
\newacronym{scifio}{SCIFIO}{SCientific Image File Input and Output}
\newacronym{scp}{SCP}{Secure CoPy}
\newacronym{sftp}{SFTP}{Secure File Transfer Protocol}
\newacronym{ssh}{SSH}{Secure SHell}
\newacronym{swt}{SWT}{Standard Widget Toolkit}
\newacronym{ui}{UI}{user interface}
\newacronym{uri}{URI}{Uniform Resource Identifier}
\newacronym{url}{URL}{Uniform Resource Locator}
\newacronym{webdav}{WebDAV}{Web Distributed Authoring and Versioning}
\newacronym{xml}{XML}{eXtensible Markup Language}

%%%%%%%%%%%%%%%%%%%%%%%%%%%%%%%%%%%%%%%%%%%%%%%%%
%%                                             %%
%%  If you wish to display your graphics for   %%
%%  your own use using includegraphic or       %%
%%  includegraphics, then comment out the      %%
%%  following two lines of code.               %%
%%  NB: These line *must* be included when     %%
%%  submitting to BMC.                         %%
%%  All figure files must be submitted as      %%
%%  separate graphics through the BMC          %%
%%  submission process, not included in the    %%
%%  submitted article.                         %%
%%                                             %%
%%%%%%%%%%%%%%%%%%%%%%%%%%%%%%%%%%%%%%%%%%%%%%%%%


\def\includegraphic{}
\def\includegraphics{}



%%% Put your definitions there:
\startlocaldefs
\endlocaldefs


%%% Begin ...
\begin{document}

%%% Start of article front matter
\begin{frontmatter}

\begin{fmbox}
\dochead{Software}

%%%%%%%%%%%%%%%%%%%%%%%%%%%%%%%%%%%%%%%%%%%%%%
%%                                          %%
%% Enter the title of your article here     %%
%%                                          %%
%%%%%%%%%%%%%%%%%%%%%%%%%%%%%%%%%%%%%%%%%%%%%%

\title{ImageJ2: ImageJ for the next generation of scientific image data}

%%%%%%%%%%%%%%%%%%%%%%%%%%%%%%%%%%%%%%%%%%%%%%
%%                                          %%
%% Enter the authors here                   %%
%%                                          %%
%% Specify information, if available,       %%
%% in the form:                             %%
%%   <key>={<id1>,<id2>}                    %%
%%   <key>=                                 %%
%% Comment or delete the keys which are     %%
%% not used. Repeat \author command as much %%
%% as required.                             %%
%%                                          %%
%%%%%%%%%%%%%%%%%%%%%%%%%%%%%%%%%%%%%%%%%%%%%%

\author[
   addressref={aff1}
]{\inits{CTR}\fnm{Curtis T} \snm{Rueden}}
\author[
   addressref={aff1,aff2}
]{\inits{JS}\fnm{Johannes} \snm{Schindelin}}
\author[
   addressref={aff1}
]{\inits{MCH}\fnm{Mark C} \snm{Hiner}}
\author[
   addressref={aff1}
]{\inits{BED}\fnm{Barry E} \snm{DeZonia}}
\author[
   addressref={aff1}
]{\inits{AEW}\fnm{Alison E} \snm{Walter}}
\author[
   addressref={aff1,aff2},
   email={eliceiri@wisc.edu}
]{\inits{KWE}\fnm{Kevin W} \snm{Eliceiri}}

%%%%%%%%%%%%%%%%%%%%%%%%%%%%%%%%%%%%%%%%%%%%%%
%%                                          %%
%% Enter the authors' addresses here        %%
%%                                          %%
%% Repeat \address commands as much as      %%
%% required.                                %%
%%                                          %%
%%%%%%%%%%%%%%%%%%%%%%%%%%%%%%%%%%%%%%%%%%%%%%

\address[id=aff1]{%                           % unique id
  \orgname{Laboratory for Optical and Computational Instrumentation, University of Wisconsin at Madison},
  \city{Madison},
  \state{Wisconsin},
  \cny{USA}
}
\address[id=aff2]{%
  \orgname{Morgridge Institute for Research},
  \city{Madison},
  \state{Wisconsin},
  \cny{USA}
}

\end{fmbox}% comment this for two column layout

%%%%%%%%%%%%%%%%%%%%%%%%%%%%%%%%%%%%%%%%%%%%%%
%%                                          %%
%% The Abstract begins here                 %%
%%                                          %%
%% Please refer to the Instructions for     %%
%% authors on http://www.biomedcentral.com  %%
%% and include the section headings         %%
%% accordingly for your article type.       %%
%%                                          %%
%%%%%%%%%%%%%%%%%%%%%%%%%%%%%%%%%%%%%%%%%%%%%%

%% The Abstract of the manuscript should not exceed 350 words and must be
%% structured into separate sections: Background, the context and purpose of
%% the study; Results, the main findings; Conclusions, brief summary and
%% potential implications.
%%
%% Please do not use abbreviations or references in the abstract. Please
%% see also our guide for writing an easily accessible abstract.

\begin{abstractbox}

\begin{abstract} % abstract
\parttitle{Background}
  ImageJ is an image analysis program extensively used in the biological
  sciences and beyond. Due to its ease of use, recordable macro language, and
  extensible plug-in architecture, ImageJ enjoys contributions from
  non-programmers, amateur programmers, and professional developers alike---a
  feat that arguably no other image analysis package, commercial or open
  source, has yet achieved. Enabling such a diversity of contributors has
  resulted in a unique community that spans the biological and physical
  sciences, making ImageJ an invaluable resource across countless disciplines.
  However, a rapidly growing user base, diverging plugin suites, and technical
  limitations have revealed a clear need for a concerted software engineering
  effort to support emerging imaging paradigms, to ensure the software's
  ability to handle the requirements of modern science.

\parttitle{Results}
  Our goal was ambitious: to create a future-proof tool without sacrificing the
  existing community. We rebuilt ImageJ from the ground up, engineering a
  powerful plugin mechanism that facilitates extensibility at every level. This
  next-generation ImageJ, called ``ImageJ2'' in places where the distinction
  matters, provides a host of new functionality. It separates concerns, fully
  decoupling the data model from the user interface. It emphasizes integration
  with external applications to maximize interoperability. Its robust new
  plugin framework allows everything from image formats, to scripting
  languages, to visualization to be extended by the community. The redesigned
  data model supports arbitrarily large, N-dimensional datasets, which are
  increasingly common in modern image acquisition. Despite the scope of these
  changes, backwards compatibility is maintained such that this new
  functionality can be seamlessly integrated with the classic ImageJ interface,
  allowing users and developers to migrate to these new methods at their own
  pace.

\parttitle{Conclusions}
  Scientific imaging benefits from open-source programs that advance new method
  development and deployment to a diverse audience. ImageJ has excelled in this
  regard; however, due to new and emerging challenges, it is at a critical
  development crossroads. The described improvements provide a framework
  adaptable to future needs, enabling continued success and innovation. Future
  efforts will focus on implementing new algorithms in this framework and
  expanding collaborations with other popular scientific software suites.
\end{abstract}

%%%%%%%%%%%%%%%%%%%%%%%%%%%%%%%%%%%%%%%%%%%%%%
%%                                          %%
%% The keywords begin here                  %%
%%                                          %%
%% Put each keyword in separate \kwd{}.     %%
%%                                          %%
%%%%%%%%%%%%%%%%%%%%%%%%%%%%%%%%%%%%%%%%%%%%%%

\begin{keyword}
\kwd{ImageJ}
\kwd{ImageJ2}
\kwd{image processing}
\kwd{N-dimensional}
\kwd{interoperability}
\kwd{extensibility}
\kwd{reproducibility}
\kwd{open source}
\kwd{open development}
\end{keyword}

% MSC classifications codes, if any
%\begin{keyword}[class=AMS]
%\kwd[Primary ]{}
%\kwd{}
%\kwd[; secondary ]{}
%\end{keyword}

\end{abstractbox}
%
%\end{fmbox}% uncomment this for twcolumn layout

\end{frontmatter}

%%%%%%%%%%%%%%%%%%%%%%%%%%%%%%%%%%%%%%%%%%%%%%
%%                                          %%
%% The Main Body begins here                %%
%%                                          %%
%% Please refer to the instructions for     %%
%% authors on:                              %%
%% http://www.biomedcentral.com/info/authors%%
%% and include the section headings         %%
%% accordingly for your article type.       %%
%%                                          %%
%% See the Results and Discussion section   %%
%% for details on how to create sub-sections%%
%%                                          %%
%% use \cite{...} to cite references        %%
%%  \cite{koon} and                         %%
%%  \cite{oreg,khar,zvai,xjon,schn,pond}    %%
%%  \nocite{smith,marg,hunn,advi,koha,mouse}%%
%%                                          %%
%%%%%%%%%%%%%%%%%%%%%%%%%%%%%%%%%%%%%%%%%%%%%%

%%%%%%%%%%%%%%%%%%%%%%%%% start of article main body
% <put your article body there>

%% The Background section should be written in a way that is accessible to
%% researchers without specialist knowledge in that area and must clearly
%% state - and, if helpful, illustrate - the background to the research and its
%% aims. It should clearly described the relevant context and the specific
%% issue which the software described is intended to address.

\section*{Background}
Text for this section.

%% This should include a description of the overall architecture of the
%% software implementation, along with details of any critical issues and
%% how they were addressed.

\section*{Implementation}
Text for this section.

%% The user interface should be described and a discussion of the intended uses
%% of the software, and the benefits that are envisioned, should be included,
%% together with data on how its performance and functionality compare with,
%% and improve, on functionally similar existing software. A case study of the
%% use of the software may be presented. The planned future development of new
%% features, if any, should be mentioned.

\section*{Results and Discussion}
Text for this section.

%% This should state clearly the main conclusions and provide an explanation of
%% the importance and relevance of the case, data, opinion, database or
%% software reported.

\section*{Conclusions}
Text for this section.

%% If abbreviations are used in the text they should be defined in the text at
%% first use, and a list of abbreviations should be provided.

\printglossary[title=List of abbreviations,type=\acronymtype,style=long]

%%%%%%%%%%%%%%%%%%%%%%%%%%%%%%%%%%%%%%%%%%%%%%
%%                                          %%
%% Backmatter begins here                   %%
%%                                          %%
%%%%%%%%%%%%%%%%%%%%%%%%%%%%%%%%%%%%%%%%%%%%%%

\begin{backmatter}

\section*{Declarations}

\subsection*{Competing interests}
  The authors declare that they have no competing interests.

\subsection*{Author's contributions}
  CTR acted as the technical lead of the ImageJ2 project and primary architect
  of ImageJ2's software architecture. JS migrated key portions of
  \acrshort{fiji} into ImageJ2, including the Launcher and Updater components,
  and advised and improved upon many architectural aspects of ImageJ2,
  particularly the legacy layer. MCH served as the lead \acrshort{scifio}
  developer and contributed to all layers of the ImageJ software stack. BED
  developed substantial portions of the ImageJ2 codebase, including much of the
  legacy layer for backwards compatibility, prototype versions of ImageJ Ops
  for numerical processing, and many command implementations. AEW contributed
  to ImageJ Ops and the ImageJ-\acrshort{omero} integration layer. Lastly, as
  the primary principal investigator of ImageJ2, KWE directed and advised on
  all aspects of the project, including development directions and priorities.
  All authors contributed to, read, and approved the final manuscript.

\subsection*{Acknowledgements}
  Text for this section \ldots

%%%%%%%%%%%%%%%%%%%%%%%%%%%%%%%%%%%
%%                               %%
%% Endnotes                      %%
%%                               %%
%%%%%%%%%%%%%%%%%%%%%%%%%%%%%%%%%%%
\section*{Endnotes}
  None.

%%%%%%%%%%%%%%%%%%%%%%%%%%%%%%%%%%%%%%%%%%%%%%%%%%%%%%%%%%%%%
%%                  The Bibliography                       %%
%%                                                         %%
%%  Bmc_mathpys.bst  will be used to                       %%
%%  create a .BBL file for submission.                     %%
%%  After submission of the .TEX file,                     %%
%%  you will be prompted to submit your .BBL file.         %%
%%                                                         %%
%%                                                         %%
%%  Note that the displayed Bibliography will not          %%
%%  necessarily be rendered by Latex exactly as specified  %%
%%  in the online Instructions for Authors.                %%
%%                                                         %%
%%%%%%%%%%%%%%%%%%%%%%%%%%%%%%%%%%%%%%%%%%%%%%%%%%%%%%%%%%%%%

% if your bibliography is in bibtex format, use those commands:
\bibliographystyle{bmc-mathphys} % Style BST file (bmc-mathphys, vancouver, spbasic).
\bibliography{imagej2}      % Bibliography file (usually '*.bib' )
% for author-year bibliography (bmc-mathphys or spbasic)
% a) write to bib file (bmc-mathphys only)
% @settings{label, options="nameyear"}
% b) uncomment next line
%\nocite{label}

% or include bibliography directly:
% \begin{thebibliography}
% \bibitem{b1}
% \end{thebibliography}

%%%%%%%%%%%%%%%%%%%%%%%%%%%%%%%%%%%
%%                               %%
%% Figures                       %%
%%                               %%
%% NB: this is for captions and  %%
%% Titles. All graphics must be  %%
%% submitted separately and NOT  %%
%% included in the Tex document  %%
%%                               %%
%%%%%%%%%%%%%%%%%%%%%%%%%%%%%%%%%%%

%%
%% Do not use \listoffigures as most will included as separate files

\section*{Figures}
  \begin{figure}[h!]
  \caption{\csentence{Sample figure title.}
      A short description of the figure content
      should go here.}
      \end{figure}

\begin{figure}[h!]
  \caption{\csentence{Sample figure title.}
      Figure legend text.}
      \end{figure}

%%%%%%%%%%%%%%%%%%%%%%%%%%%%%%%%%%%
%%                               %%
%% Tables                        %%
%%                               %%
%%%%%%%%%%%%%%%%%%%%%%%%%%%%%%%%%%%

%% Use of \listoftables is discouraged.
%%
\section*{Tables}
\begin{table}[h!]
\caption{Sample table title. This is where the description of the table should go.}
      \begin{tabular}{cccc}
        \hline
           & B1  &B2   & B3\\ \hline
        A1 & 0.1 & 0.2 & 0.3\\
        A2 & ... & ..  & .\\
        A3 & ..  & .   & .\\ \hline
      \end{tabular}
\end{table}

%%%%%%%%%%%%%%%%%%%%%%%%%%%%%%%%%%%
%%                               %%
%% Additional Files              %%
%%                               %%
%%%%%%%%%%%%%%%%%%%%%%%%%%%%%%%%%%%

\section*{Additional Files}
  \subsection*{Additional file 1 --- Sample additional file title}
    Additional file descriptions text (including details of how to
    view the file, if it is in a non-standard format or the file extension).  This might
    refer to a multi-page table or a figure.

  \subsection*{Additional file 2 --- Sample additional file title}
    Additional file descriptions text.


\end{backmatter}
\end{document}
