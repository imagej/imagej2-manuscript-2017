%% BioMed_Central_Tex_Template_v1.06

%%IMPORTANT: do not delete the first line of this template
%%It must be present to enable the BMC Submission system to
%%recognise this template!!

%%%%%%%%%%%%%%%%%%%%%%%%%%%%%%%%%%%%%%%%%
%%                                     %%
%%  LaTeX template for BioMed Central  %%
%%     journal article submissions     %%
%%                                     %%
%%          <8 June 2012>              %%
%%                                     %%
%%                                     %%
%%%%%%%%%%%%%%%%%%%%%%%%%%%%%%%%%%%%%%%%%


%%%%%%%%%%%%%%%%%%%%%%%%%%%%%%%%%%%%%%%%%%%%%%%%%%%%%%%%%%%%%%%%%%%%%
%%                                                                 %%
%% For instructions on how to fill out this Tex template           %%
%% document please refer to Readme.html and the instructions for   %%
%% authors page on the biomed central website                      %%
%% http://www.biomedcentral.com/info/authors/                      %%
%%                                                                 %%
%% Please do not use \input{...} to include other tex files.       %%
%% Submit your LaTeX manuscript as one .tex document.              %%
%%                                                                 %%
%% All additional figures and files should be attached             %%
%% separately and not embedded in the \TeX\ document itself.       %%
%%                                                                 %%
%% BioMed Central currently use the MikTex distribution of         %%
%% TeX for Windows) of TeX and LaTeX.  This is available from      %%
%% http://www.miktex.org                                           %%
%%                                                                 %%
%%%%%%%%%%%%%%%%%%%%%%%%%%%%%%%%%%%%%%%%%%%%%%%%%%%%%%%%%%%%%%%%%%%%%

%%% additional documentclass options:
%  [doublespacing]
%  [linenumbers]   - put the line numbers on margins

%%% loading packages, author definitions

%\documentclass[twocolumn]{bmcart}% uncomment this for twocolumn layout and comment line below
\documentclass{bmcart}

%%% Load packages
%\usepackage{amsthm,amsmath}
%\RequirePackage{natbib}
%\RequirePackage[authoryear]{natbib}% uncomment this for author-year bibliography
%\RequirePackage{hyperref}
\usepackage[utf8]{inputenc} %unicode support
%\usepackage[applemac]{inputenc} %applemac support if unicode package fails
%\usepackage[latin1]{inputenc} %UNIX support if unicode package fails
\usepackage{placeins} %for FloatBarrier hack
\usepackage{multirow} %for multirow table cells
\usepackage{enumitem} %for itemize leftmargin

%% We use the glossaries package for the list of acronyms.
\usepackage[nonumberlist]{glossaries}
\renewcommand{\glossarypreamble}{\footnotesize}
\makeglossaries
\newacronym{alida}{Alida}{Advanced Library for Integrated Development of data analysis Applications}
\newacronym{api}{API}{Application Programming Interface}
\newacronym{awt}{AWT}{Abstract Windowing Toolkit}
\newacronym{bar}{BAR}{Broadly Applicable Routines}
\newacronym{bmc}{BMC}{BioMed Central}
\newacronym{bsd}{BSD}{Berkeley Software Distribution}
\newacronym{ci}{CI}{continuous integration}
\newacronym{cpu}{CPU}{Central Processing Unit}
\newacronym{di}{DI}{dependency injection}
\newacronym{dry}{DRY}{Don't Repeat Yourself}
\newacronym{epfl}{EPFL}{École Polytechnique Fédérale de Lausanne}
\newacronym{fiji}{Fiji}{Fiji Is Just ImageJ}
\newacronym{gimp}{GIMP}{GNU Image Manipulation Program}
\newacronym{gnu}{GNU}{GNU's Not Unix}
\newacronym{gpu}{GPU}{Graphics Processing Unit}
\newacronym{gtk}{GTK}{GIMP ToolKit}
\newacronym{http}{HTTP}{Hypertext Transfer Protocol}
\newacronym{io}{I/O}{Input/Output}
\newacronym{ide}{IDE}{Integrated Development Environment}
\newacronym{ioc}{IoC}{inversion of control}
\newacronym{itk}{ITK}{Insight ToolKit}
\newacronym{jar}{JAR}{Java ARchive}
\newacronym{jit}{JIT}{Just-In-Time compiler}
\newacronym{jvm}{JVM}{Java Virtual Machine}
\newacronym{knime}{KNIME}{KoNstanz Information MinEr}
\newacronym{knip}{KNIP}{KNIME Image Processing}
\newacronym{laf}{L\&F}{Look \& Feel}
\newacronym{list}{LIST}{Luxembourg Institute of Science and Technology}
\newacronym{lut}{LUT}{color lookup table}
\newacronym{loci}{LOCI}{Laboratory for Optical and Computational Instrumentation}
\newacronym{macos}{macOS}{Macintosh Operating System}
\newacronym{matlab}{MATLAB}{MATrix LABoratory}
\newacronym{mitobo}{MiToBo}{Microscopy image analysis ToolBox}
\newacronym{nigms}{NIGMS}{National Institute of General Medical Sciences}
\newacronym{nih}{NIH}{National Institutes of Health}
\newacronym{ome}{OME}{Open Microscopy Environment}
\newacronym{omero}{OMERO}{OME Remote Objects}
\newacronym{opencv}{OpenCV}{Open source Computer Vision library}
\newacronym{pom}{POM}{Project Object Model}
\newacronym{ram}{RAM}{Random-Access Memory}
\newacronym{rest}{RESTful}{REpresentational State Transfer}
\newacronym{rgb}{RGB}{Red+Green+Blue color model}
\newacronym[longplural={regions of interest}]{roi}{ROI}{region of interest}
\newacronym{scifio}{SCIFIO}{SCientific Image Format Input and Output}
\newacronym{scp}{SCP}{Secure CoPy}
\newacronym{sftp}{SFTP}{Secure File Transfer Protocol}
\newacronym{ssh}{SSH}{Secure SHell}
\newacronym{swt}{SWT}{Standard Widget Toolkit}
\newacronym{ui}{UI}{user interface}
\newacronym{uri}{URI}{Uniform Resource Identifier}
\newacronym{url}{URL}{Uniform Resource Locator}
\newacronym{webdav}{WebDAV}{Web Distributed Authoring and Versioning}
\newacronym{xml}{XML}{eXtensible Markup Language}

%%%%%%%%%%%%%%%%%%%%%%%%%%%%%%%%%%%%%%%%%%%%%%%%%
%%                                             %%
%%  If you wish to display your graphics for   %%
%%  your own use using includegraphic or       %%
%%  includegraphics, then comment out the      %%
%%  following two lines of code.               %%
%%  NB: These line *must* be included when     %%
%%  submitting to BMC.                         %%
%%  All figure files must be submitted as      %%
%%  separate graphics through the BMC          %%
%%  submission process, not included in the    %%
%%  submitted article.                         %%
%%                                             %%
%%%%%%%%%%%%%%%%%%%%%%%%%%%%%%%%%%%%%%%%%%%%%%%%%


\def\includegraphic{}
\def\includegraphics{}



%%% Put your definitions there:
\startlocaldefs
\endlocaldefs

\renewcommand{\theenumi}{\arabic{enumi}.} %include period in numbered lists

%%% Begin ...
\begin{document}

%%% Start of article front matter
\begin{frontmatter}

\begin{fmbox}
\dochead{Software}

%%%%%%%%%%%%%%%%%%%%%%%%%%%%%%%%%%%%%%%%%%%%%%
%%                                          %%
%% Enter the title of your article here     %%
%%                                          %%
%%%%%%%%%%%%%%%%%%%%%%%%%%%%%%%%%%%%%%%%%%%%%%

\title{ImageJ2: ImageJ for the next generation of scientific image data}

%%%%%%%%%%%%%%%%%%%%%%%%%%%%%%%%%%%%%%%%%%%%%%
%%                                          %%
%% Enter the authors here                   %%
%%                                          %%
%% Specify information, if available,       %%
%% in the form:                             %%
%%   <key>={<id1>,<id2>}                    %%
%%   <key>=                                 %%
%% Comment or delete the keys which are     %%
%% not used. Repeat \author command as much %%
%% as required.                             %%
%%                                          %%
%%%%%%%%%%%%%%%%%%%%%%%%%%%%%%%%%%%%%%%%%%%%%%

\author[
   addressref={aff1}
]{\inits{CTR}\fnm{Curtis T} \snm{Rueden}}
\author[
   addressref={aff1,aff2}
]{\inits{JS}\fnm{Johannes} \snm{Schindelin}}
\author[
   addressref={aff1}
]{\inits{MCH}\fnm{Mark C} \snm{Hiner}}
\author[
   addressref={aff1}
]{\inits{BED}\fnm{Barry E} \snm{DeZonia}}
\author[
   addressref={aff1,aff2}
]{\inits{AEW}\fnm{Alison E} \snm{Walter}}
\author[
   addressref={aff1,aff2}
]{\inits{ETA}\fnm{Ellen T} \snm{Arena}}
\author[
   addressref={aff1,aff2},
   email={eliceiri@wisc.edu}
]{\inits{KWE}\fnm{Kevin W} \snm{Eliceiri}}

%%%%%%%%%%%%%%%%%%%%%%%%%%%%%%%%%%%%%%%%%%%%%%
%%                                          %%
%% Enter the authors' addresses here        %%
%%                                          %%
%% Repeat \address commands as much as      %%
%% required.                                %%
%%                                          %%
%%%%%%%%%%%%%%%%%%%%%%%%%%%%%%%%%%%%%%%%%%%%%%

\address[id=aff1]{%                           % unique id
  \orgname{Laboratory for Optical and Computational Instrumentation, University of Wisconsin at Madison},
  \city{Madison},
  \state{Wisconsin},
  \cny{USA}
}
\address[id=aff2]{%
  \orgname{Morgridge Institute for Research},
  \city{Madison},
  \state{Wisconsin},
  \cny{USA}
}

\end{fmbox}% comment this for two column layout

%%%%%%%%%%%%%%%%%%%%%%%%%%%%%%%%%%%%%%%%%%%%%%
%%                                          %%
%% The Abstract begins here                 %%
%%                                          %%
%% Please refer to the Instructions for     %%
%% authors on http://www.biomedcentral.com  %%
%% and include the section headings         %%
%% accordingly for your article type.       %%
%%                                          %%
%%%%%%%%%%%%%%%%%%%%%%%%%%%%%%%%%%%%%%%%%%%%%%

%% The Abstract of the manuscript should not exceed 350 words and must be
%% structured into separate sections: Background, the context and purpose of
%% the study; Results, the main findings; Conclusions, brief summary and
%% potential implications.
%%
%% Please do not use abbreviations or references in the abstract. Please
%% see also our guide for writing an easily accessible abstract.

\begin{abstractbox}

\begin{abstract} % abstract
\parttitle{Background}
  ImageJ is an image analysis program extensively used in the biological
  sciences and beyond. Due to its ease of use, recordable macro language, and
  extensible plug-in architecture, ImageJ enjoys contributions from
  non-programmers, amateur programmers, and professional developers alike.
  Enabling such a diversity of contributors has resulted in a large community
  that spans the biological and physical sciences.
  However, a rapidly growing user base, diverging plugin suites, and technical
  limitations have revealed a clear need for a concerted software engineering
  effort to support emerging imaging paradigms, to ensure the software's
  ability to handle the requirements of modern science.

\parttitle{Results}
  We rewrote the entire ImageJ codebase, engineering a redesigned plugin
  mechanism intended to facilitate extensibility at every level, with the goal
  of creating a more powerful tool that continues to serve the existing
  community while addressing a wider range of scientific requirements. This
  next-generation ImageJ, called ``ImageJ2'' in places where the distinction
  matters, provides a host of new functionality. It separates concerns, fully
  decoupling the data model from the user interface. It emphasizes integration
  with external applications to maximize interoperability. Its robust new
  plugin framework allows everything from image formats, to scripting
  languages, to visualization to be extended by the community. The redesigned
  data model supports arbitrarily large, N-dimensional datasets, which are
  increasingly common in modern image acquisition. Despite the scope of these
  changes, backwards compatibility is maintained such that this new
  functionality can be seamlessly integrated with the classic ImageJ interface,
  allowing users and developers to migrate to these new methods at their own
  pace.

\parttitle{Conclusions}
  Scientific imaging benefits from open-source programs that advance new method
  development and deployment to a diverse audience. ImageJ has continuously
  evolved with this idea in mind; however, new and emerging scientific
  requirements have posed corresponding challenges for ImageJ's development.
  The described improvements provide a framework engineered for flexibility,
  intended to support these requirements as well as accommodate future needs.
  Future efforts will focus on implementing new algorithms in this framework
  and expanding collaborations with other popular scientific software suites.
\end{abstract}

%%%%%%%%%%%%%%%%%%%%%%%%%%%%%%%%%%%%%%%%%%%%%%
%%                                          %%
%% The keywords begin here                  %%
%%                                          %%
%% Put each keyword in separate \kwd{}.     %%
%%                                          %%
%%%%%%%%%%%%%%%%%%%%%%%%%%%%%%%%%%%%%%%%%%%%%%

\begin{keyword}
\kwd{ImageJ}
\kwd{ImageJ2}
\kwd{image processing}
\kwd{N-dimensional}
\kwd{interoperability}
\kwd{extensibility}
\kwd{reproducibility}
\kwd{open source}
\kwd{open development}
\end{keyword}

% MSC classifications codes, if any
%\begin{keyword}[class=AMS]
%\kwd[Primary ]{}
%\kwd{}
%\kwd[; secondary ]{}
%\end{keyword}

\end{abstractbox}
%
%\end{fmbox}% uncomment this for twcolumn layout

\end{frontmatter}

%%%%%%%%%%%%%%%%%%%%%%%%%%%%%%%%%%%%%%%%%%%%%%
%%                                          %%
%% The Main Body begins here                %%
%%                                          %%
%% Please refer to the instructions for     %%
%% authors on:                              %%
%% http://www.biomedcentral.com/info/authors%%
%% and include the section headings         %%
%% accordingly for your article type.       %%
%%                                          %%
%% See the Results and Discussion section   %%
%% for details on how to create sub-sections%%
%%                                          %%
%% use \cite{...} to cite references        %%
%%  \cite{koon} and                         %%
%%  \cite{oreg,khar,zvai,xjon,schn,pond}    %%
%%  \nocite{smith,marg,hunn,advi,koha,mouse}%%
%%                                          %%
%%%%%%%%%%%%%%%%%%%%%%%%%%%%%%%%%%%%%%%%%%%%%%

%%%%%%%%%%%%%%%%%%%%%%%%% start of article main body
% <put your article body there>

%% The Background section should be written in a way that is accessible to
%% researchers without specialist knowledge in that area and must clearly
%% state - and, if helpful, illustrate - the background to the research and its
%% aims. It should clearly described the relevant context and the specific
%% issue which the software described is intended to address.

\section*{Background}
ImageJ \cite{imagej_history} is a powerful, oft-referenced platform for image
processing, developed by Wayne Rasband at the \acrfull{nih}. Since its initial
release in 1997, ImageJ has proven paramount in many scientific endeavors and
projects, particularly those within the life sciences \cite{imagej_review}.
Over the past nineteen years, the program has evolved far beyond its originally
intended scope. After such an extended period of sustained growth, any software
project benefits from a subsequent period of scrutiny and refactoring; ImageJ
is no exception. Such restructuring helps the program to remain accessible to
newcomers, powerful enough for experts, and relevant to the demands of its
ever-growing community. As such, we have developed ImageJ2: a total redesign of
the previous incarnation (hereafter ``ImageJ 1.x''), which builds on the
original's successful qualities while improving its core architecture to
encompass the scientific demands of the decades to come. Key motivations for
the development of ImageJ2 include:

\begin{enumerate}
  \item \textbf{Supporting the next generation of image data.} Over time, the
    infrastructure of image acquisition has grown in sophistication and
    complexity. For example, in the field of microscopy we were once limited to
    single image planes. However, with modern techniques we can record much
    more information: physical location in time and space (X, Y, Z, time),
    lifetime histograms across a range of spectral emission channels,
    polarization state of light, phase and frequency, angles of rotation (e.g.,
    in light sheet fluorescence microscopy), and high-throughput screens, just
    to name a few. The ImageJ infrastructure needed improvement to work
    effectively with these new modes of image data.

  \item \textbf{Enabling new software collaborations.} The field of software
    engineering has seen an explosion of available development tools and
    infrastructure, and it is no longer realistic to expect a single standalone
    application to remain universally relevant. We wanted to improve ImageJ's
    modularity to facilitate its use as a software library, the creation of
    additional user interfaces, and integration and interoperability with
    external software suites.

  \item \textbf{Broadening the ImageJ community.} Though initially developed
    for the life sciences, ImageJ is used in various other scientific
    disciplines as well. It has the potential to be a powerful tool for any
    field that benefits from image visualization, processing, and analysis:
    earth sciences, astronomy, fluid dynamics, computer vision, signal
    processing, etc. We wanted to enhance ImageJ's impact in the greater
    scientific community by adopting software engineering best practices,
    generalizing the codebase, and providing unified, comprehensive,
    consistently structured, community-editable online resources.
\end{enumerate}

From these motivations emerge the six pillars of the ImageJ2 mission
statement:

\begin{itemize}
  \item \textbf{Design} the next generation of ImageJ, driven by the needs of
    the community.
  \item \textbf{Collaborate} across organizations, fostering open development
    through sharing and reuse.
  \item \textbf{Broaden} ImageJ's usefulness and relevance across many
    disciplines of the scientific community.
  \item \textbf{Maintain} backwards compatibility with existing ImageJ
    functionality.
  \item \textbf{Unify} online resources to a central location for the ImageJ
    community.
  \item \textbf{Lead} ImageJ development with a clear vision.
\end{itemize}

\subsection*{Why ImageJ?}
Any time a development effort of this scale is undertaken on an existing tool,
it is worth evaluating its impact and the decision to invest such resources.
The bioimage informatics field \cite{bioimage_informatics} is fortunate to have
a wide range of software tools available in both commercial and open source
arenas \cite{bioimaging_software_review}. Open-source tools are especially
important in science due to their transparency and inherent ability for sharing
and extensibility \cite{bioimaging_cell_biology}. This need and ability for
method sharing has resulted in a plethora of open-source solutions in bioimage
informatics, ranging from image acquisition tools such as $\mu$Manager
\cite{micro_manager_2010, micro_manager_2014}; databases such as Bisque
\cite{bisque} and \acrfull{omero} \cite{omero}; image analysis suites such as
Icy \cite{icy} and BioImageXD \cite{bioimagexd}; scientific workflow and
pipeline tools such as CellProfiler \cite{cellprofiler, cellprofiler_2011},
\acrshort{knime} \cite{knime, knip} and Pipeline Pilot \cite{workflow_systems};
and 3D rendering applications such as FluoRender \cite{fluorender} and Vaa3D
\cite{vaa3d}. There are many other open, bioimaging-oriented software packages
besides these, including solutions written in powerful scripting platforms such
as R, Python and MATLAB. With such an extensive array of tools, does it make
sense to invest in an updated ImageJ platform, rather than building on some
combination of more recent tools?

Actually, the ImageJ2 project does both: it rearchitects ImageJ as a shared
platform for integration and interoperability across many bioimaging software
packages. ImageJ has a unique niche in that it is not a monolithic or
single-purpose application, but instead is designed as a platform for discovery
where the bench biologist can adapt and deploy new image analysis methods. What
makes ImageJ great is not only a set of pre-designed, distributed tools
developed for a single purpose and regularly maintained and updated, but also
its powerful yet approachable plugin and macro environments that have enabled
hundreds of groups to generate results through the development of thousands of
customized plugins and scripts \cite{imagej_review, imagej_ecosystem,
imagej_list_of_update_sites}. It is this ability for sharing, and the desire to
engage the professional and amateur developer alike, that drove the development
for ImageJ2. The new version of ImageJ is a platform for extensibility and
cross-application cooperation, broadening the scope of ImageJ into a new effort
called SciJava \cite{scijava}: a collaboration of projects striving to
cooperate and build on one another both socially and technically. It is our
intent that with the developments detailed in this paper, the synergy between
these tools, which include ImageJ, \acrshort{knime}, CellProfiler,
\acrshort{omero} and others, will only increase as each tool continues to
develop, benefiting not only existing users, but new users and communities as
well.

%% The Design Goals section discusses ImageJ2's _goals_ only, not the
%% _reality_ of what it does. So it uses language like "must" and "should"
%% and "strives to" rather than "does" and "is". The latter (especially
%% details thereof) is for Implementation and Results below only.

\subsection*{Design Goals}
The central technical design goals of ImageJ2 can be divided into seven key
categories: functionality, extensibility, reproducibility, usability,
performance, compatibility and community. In this section, we discuss the goals
of ImageJ2 from its outset; for how these goals have been met in practice, see
the subsequent sections.

\subsubsection*{Functionality}
The overriding principle of ImageJ2 is to create \textbf{\textit{powerful}}
software, capable of meeting the expanding requirements of an ever-more-complex
landscape of scientific image processing and analysis for the foreseeable
future. As such, ImageJ needs to be more than just an application: it must be a
\textbf{\textit{modular}}, multi-layered set of functions with each layer
encapsulated and building upon lower layers. In computer science terminology,
ImageJ2 strives to have a proper \textbf{\textit{separation of concerns}}
between data model and display thereof, enabling use within a wide variety of
scenarios, such as headless operation---i.e., running remotely on a server,
cluster or cloud without a graphical \acrfull{ui}.

At its core, ImageJ2 aims to provide robust support for
\textbf{\textit{N-dimensional}} image data, to support domains with dimensions
beyond time and space. Examples include: multispectral and hyperspectral
images, fluorescence lifetime measured in the time or frequency domains,
multi-angle data from acquisition modalities such as light sheet fluorescence
microscopy, multi-position data from High Content Screens, and experiments
using polarized light. In general, the design must be robust enough to express
any newly emerging modalities within its infrastructure.

Finally, it is not sufficient to provide a modular framework---ImageJ2 must
also provide \textbf{\textit{built-in routines}} as default behavior for
standard tasks in image processing and analysis. These core plugins must span a
wealth of algorithms for image processing and analysis, image visualization,
and image file import and export. Such built-in features ensure users have an
application they can apply out-of-the-box.

\subsubsection*{Extensibility}
The quality that makes ImageJ most powerful---its greatest strength---is its
\textbf{\textit{extensibility}}. From its inception \cite{imagej_history},
ImageJ 1.x has had a mechanism by which users can develop their own plugins and
macros to extend its capabilities. Two decades later, a plethora of such
plugins and macros have been shared and published \cite{imagej_ecosystem}. It
is paramount that ImageJ2 maintains this ease of modification and extension by
its user community, and furthermore leverages its improved separation of
concerns to actually make user extension easier and more powerful; e.g., if
image processing plugins are agnostic to user interface, new interfaces can be
developed without a loss of functionality.

A related preeminent concern is \textbf{\textit{interoperability}}. There is no
silver bullet in image processing. No matter how powerful ImageJ becomes or how
many extensions exist, there will always be powerful and useful alternative
tools available. Users benefit most when information can easily be exchanged
between such tools. One of ImageJ2's primary motivations is to enable usage of
ImageJ code from other applications and toolkits, and to support open standards
for data storage and exchange.

\subsubsection*{Reproducibility}
For ImageJ to be truly useful to the scientific community, it must be not only
technically feasible to extend, but also socially feasible, without legal
obstacles or other restrictions preventing the free exchange of scientific
ideas. To that end, ImageJ must be not only open source, but offer full
\textbf{\textit{reproducibility}}, following an \textbf{\textit{open
development process}} which we believe is an optimal fit for open scientific
inquiry \cite{software_usability}. We want to enable the community to not just
use ImageJ, but also to build upon it, with all project resources---revision
history, project roadmap, community contribution process, etc.---publicly
accessible, and development discussions taking place in public, archived
communication channels so that interested parties can remain informed of and
contribute to the project's future directions. Such transparency also
facilitates sensible, defensible software development processes and fosters
responsibility amongst those involved in the ImageJ project.

\subsubsection*{Usability}
Modular systems composed of many components often have a corresponding increase
in conceptual complexity, making them harder to understand and use. To avoid
this pitfall, ImageJ2 employs the idea of complexity minimization: seeking
\textbf{\textit{sensible defaults}} that make simple things easy, but difficult
things still possible. The lowest-level software layers should define the
program's full power, while each subsequent layer reduces visible complexity by
choosing default parameters suitable for common tasks. The highest levels
should provide users with the simplicity of a ``big green button,'' performing
the most commonly desired tasks with ease---the powerful inner machinery
remaining unseen, yet accessible when needed.

To bridge the gap between extensibility and usability, there must be a painless
process of installing new functionality: a built-in, configurable
\textbf{\textit{automatic update mechanism}} to manage extensions and keep the
software up-to-date. This update mechanism must be scalable and distributed,
such that software developers can publish their own extensions on their own
websites, without needing to obtain permission from a central authority.

\subsubsection*{Performance}
N-dimensional images and the ever-expanding size of datasets increase the
computation requirements placed on analysis routines. For ImageJ2 to succeed,
it must accomplish its goals without negatively impacting performance
\textbf{\textit{efficiency}} in time---e.g., \acrfull{cpu} and
\acrfull{gpu}---or space---e.g., \acrfull{ram} and disk. Furthermore, to ensure
ImageJ2 meets performance needs for a wide variety of use cases, it should
offer choices surrounding usage of available resources, as well as sensible
defaults for balancing performance in common scenarios.

Another key consideration for performance is \textbf{\textit{scalability}}:
ImageJ must be capable of operating on increasingly huge datasets. In cloud
computing, this requirement is often met via elasticity: the ability to
transparently provision additional computing resources---i.e., throw more
computers at the problem \cite{hardware_is_cheap}. We are at the dawn of the
``Big Data'' era of computing, where both computation and storage are scalable
resources which can be purchased from remote server farms. Software like ImageJ
which hopes to remain effective for serious scientific inquiry into the coming
decades must be architected so that its algorithms scale well to increasingly
large data processed in parallel across increasingly large numbers of
\acrshort{cpu} and \acrshort{gpu} cores.

\subsubsection*{Compatibility}
There are a vast number of existing extensions---plugins, macros, and
scripts---for the original ImageJ 1.x application which have proven extremely
useful to the user community \cite{imagej_ecosystem}. ImageJ2 must continue to
support these extensions as faithfully as possible, while also providing a
clear incremental migration path to take advantage of the new framework.

\subsubsection*{Community}
The principal non-technical goal of ImageJ2 is to serve the ImageJ community as
it evolves and grows; to that end, several community-oriented technical goals
naturally follow. The ImageJ project must provide \textbf{\textit{unified
online resources}} including a central community-editable website, discussion
forum, and online technical resources for managing community extensions of
ImageJ. And the ImageJ application itself must work in concert with these
resources---e.g., users should be able to report bugs directly to online issue
tracking systems when something goes wrong.

%% This should include a description of the overall architecture of the
%% software implementation, along with details of any critical issues and
%% how they were addressed.

\section*{Implementation}
Broadly speaking, ImageJ2 components are classified into one of four domains:

\begin{itemize}
  \item \textbf{SciJava.} The most fundamental layers of ImageJ2 are
    independent from image processing, but rather provide needed functionality
    common to many applications. On a technical level, the SciJava core
    components are a set of standard Java libraries for managing extensible
    applications. Socially, the SciJava initiative is a pledge among
    cooperating organizations to work together, reuse code, and synergize
    wherever possible \cite{imagej_scijava}.
  \item \textbf{ImgLib2.} To ensure generality of image analysis, ImageJ2 is
    built on the flexible ImgLib2 container model \cite{imglib2}. Decoupling the
    elements of image representation, ImgLib2 components enable general image
    processing algorithms to be reused, regardless of image type, source, or
    organization.
  \item \textbf{\acrfull{scifio}.} \acrshort{scifio} components define
    standards for reading, writing, and translating between image formats
    \cite{scifio}. These libraries ensure a broad spectrum of image data can be
    interpreted in a consistent manner across all SciJava applications.
  \item \textbf{ImageJ.} Low-level components establish image metadata and algorithm
    patterns, built on the SciJava and ImgLib2 layers. High-level
    components focus on ``end user'' tools for working with image data,
    and include user interfaces, user-facing commands, and the top-level
    ImageJ application \cite{imagej_web_site}.
\end{itemize}

These layers, taken as a whole, form the \textbf{ImageJ software stack}
\cite{imagej_architecture}, the core set of components upon which ImageJ-based
applications are built.

Each domain is itself divided into many individual libraries, each of which
targets a particular function. This separation of concerns provides a logical
organization which allows targeted reuse and extension of any given
functionality of interest.

The following sections describe, in order from lowest to highest level, the
essential backbone libraries of ImageJ2. Note that this is not an exhaustive
list of components, as many components across these domains provide secondary
functions---e.g.: script languages, build management, \acrshort{ui} elements,
or targeted implementations of specific features.

\subsection*{SciJava Common}
The ground floor of the ImageJ software stack is the SciJava Common library
\cite{imagej_sjc}, providing the core framework for creating extensible
applications. The heart of SciJava Common is its \textbf{application
container}, the \texttt{Context} class. Each \texttt{Context} encapsulates
runtime application state: available extensions, open images and documents,
user settings, etc. The application container paradigm allows multiple
independently configured instances of SciJava applications to run concurrently
within the same \acrfull{jvm}.

\subsubsection*{Service framework}
The application container consists of a collection of \textbf{services}, which
are initialized dynamically at runtime. These services provide methods which
operate on the system in various ways, such as opening data, manipulating
images, or displaying user interface elements on screen. Taken as a whole,
these service methods constitute the bulk of the \acrfull{api} of ImageJ.
Software developers are free to extend the system with new needed services
and/or override any aspect of behavior provided by existing services. This
approach is in contrast to the most common naive design of many software
projects, which use global ``static'' state and functions, whose behavior is
difficult or impossible to override or enhance in downstream code.

The SciJava Common library itself provides the most fundamental of these
services, such as:

\begin{itemize}
  \item A \textbf{plugin service}, which dynamically discovers available
    plugins using an index generated at compile time by a Java annotation
    processor. This plugin index is used to bootstrap the application context,
    since services are themselves a type of plugin.
  \item An \textbf{event service}, which provides a hierarchical
    publish/subscribe model for event handling.
  \item A \textbf{log service}, for environment-agnostic data logging.
  \item An \textbf{object service}, which keeps a central typed index of
    available objects.
  \item A \textbf{thread service}, which manages a thread pool and dispatch
    thread(s) for code execution.
  \item An \textbf{I/O service}, for reading and writing of data.
  \item A \textbf{preference service}, for saving and restoring user-specific
    preferences.
\end{itemize}

In principle, SciJava Common is similar to frameworks such as Spring
\cite{spring}, offering standard software engineering patterns such as
\acrfull{di} \cite{dependency_injection} and \acrfull{ioc} \cite{ioc}, but
tailored to the needs of collaborative scientific projects like ImageJ. For
example, SciJava Common provides a generalized I/O mechanism for opening data
from any source, but the library itself has no specific knowledge of how to
open \acrshort{xml} documents, microscopy image formats, or spreadsheets of
numerical results---such functionality is provided by downstream components as
SciJava plugins (see next section).

\subsubsection*{Plugin framework}
SciJava Common provides a unified mechanism for defining \textbf{plugins}:
extensions which add new features or behavior to the software, and/or modify
existing behavior. Plugins are discovered by the system at runtime, and ordered
according to assigned priorities and types, forming type hierarchies:
structural trees that define how each individual plugin fits into the system.
The typical pattern for a desired sort of functionality is to define a
dedicated plugin type, then implement plugins that fulfill that operation in
various ways. SciJava Common is designed to make virtually any aspect of an
application extensible. Some of the most critical plugin categories and types
include:

\paragraph*{Core extensibility}
\begin{itemize}
  \item \textbf{\texttt{Service}} -- A collection of related functionality,
    expressed as an \acrfull{api}. SciJava services are singletons with respect
    to each application context. For example, each instance of ImageJ2 has
    exactly one \texttt{AnimationService} responsible for managing animations,
    with methods to start and stop animations, select the dimension over which
    to animate, adjust frame rate, and other options. Note that while the
    behavior of services can certainly be modified by extensions, doing so is
    primarily the domain of advanced developers looking to radically alter
    behavior of the system.
  \item \textbf{\texttt{IOPlugin}} -- A plugin that reads data from and/or
    writes data to a location, such as a file on disk. For example, the SciJava
    layer provides I/O plugins for common text formats such as Markdown
    \cite{markdown}, while the \acrshort{scifio} layer provides an I/O plugin
    for image formats.
\end{itemize}

\paragraph*{Modules}
\begin{itemize}
  \item \textbf{\texttt{Command}} -- An operation, more generally known as a
    SciJava \textbf{\textit{module}}, with typed inputs and outputs. These
    modules typically appear in the menu system of the application's user
    interface, but can be exposed via interoperability mechanisms in many other
    ways, such as nodes in \acrshort{knime} or modules in CellProfiler
    \cite{cellprofiler}. When ImageJ users talk about ``writing a plugin'' they
    usually mean a \texttt{Command}. See ``Module framework'' below for more on
    SciJava modules.
  \item \textbf{\texttt{ScriptLanguage}} -- A programming language for SciJava
    scripts. Each script language plugin provides the logic necessary to
    execute scripts written in that language (e.g., JavaScript or Python) as
    SciJava modules with typed inputs and outputs, in a similar way to
    commands. It also makes it possible to express operations as code snippets
    that can be reused in scripts to repeat those operations.
  \item \textbf{\texttt{Converter}} -- A plugin which transforms data from one
    type of object to a different type of object. Converters greatly extend the
    concept of type conversion from what Java provides out of the box to
    provide automatic conversion in a wide and extensible set of circumstances.
    For example, it becomes possible for an algorithm to accept a string in
    place of a floating point numerical value, as long as that string can be
    parsed to such a value---or to transparently convert between
    normally-incompatible image data structures from different image processing
    ecosystems.
  \item \textbf{\texttt{ModulePreprocessor}} -- A ``meta-module'' which
    prepares modules to run. For example, the \texttt{LoadInputsPreprocessor}
    populates a module's inputs with the last-used values as defaults, which
    can save the user a lot of time. Preprocessor plugins are executed in
    priority order as part of the module ``preprocessing chain'' before the
    module is actually executed.
  \item \textbf{\texttt{ModulePostprocessor}} -- A ``meta-module'' which does
    something with a module after it has run. For example, the
    \texttt{DisplayPostprocessor} takes care of displaying the outputs of a
    module after it has completed execution. Postprocessor plugins are executed
    in priority order as part of the module ``postprocessing chain'' after the
    module is actually executed.
\end{itemize}

\paragraph*{User interface}
\begin{itemize}
  \item \textbf{\texttt{UserInterface}} -- A plugin providing an application
    \acrshort{ui}. These plugins include functionality for creating and showing
    windows and dialogs. ImageJ2 includes a user interface written in Java's
    Swing toolkit which is modeled closely after the ImageJ 1.x design, as well
    as a \texttt{UserInterface} plugin that wraps ImageJ 1.x itself. But other
    \acrshortpl{ui} are equally possible; since a \acrshort{ui} is simply a
    type of plugin, anyone can develop their own SciJava \acrshort{ui} without
    any code changes to the core system. The system is even flexible enough to
    display multiple \acrshortpl{ui} simultaneously.
  \item \textbf{\texttt{Platform}} -- A plugin which enables customization of
    behavior based on machine-specific criteria, such as specific flavor of
    operating system or Java language, including type, architecture, or
    version. For example, on \acrshort{macos}, the menu bar appears at the top
    of the screen, with the About, Preferences, and Quit commands relocated to
    the Application menu.
  \item \textbf{\texttt{InputWidget}} -- A user interface element for
    harvesting typed inputs. Typically, these widgets are presented as part of
    a form in a dialog box which prompts the user to fill in input values of a
    module. In principle, the widgets can be used for anything requiring typed
    input from the user. For example, a \texttt{FileWidget} allows the user to
    select a file (\texttt{java.io.File}) on disk, while a ToggleWidget
    provides a boolean toggle (typically rendered as a checkbox). The SciJava
    layer provides \acrshort{ui}-agnostic interfaces to the common widget
    types, along with widget implementations corresponding to each supported
    \texttt{UserInterface} plugin. However, an extension to the system can not
    only implement its own data structure classes which it uses as inputs to
    its modules; it can also provide corresponding widgets for those
    structures, allowing the user to populate them from the user interface in
    innovative ways.
  \item \textbf{\texttt{Display}} -- A plugin for visualizing data. For
    example, an ImageJ2 \texttt{ImageDisplay} can show two-dimensional planes
    of N-dimensional image data in a window with sliders for controlling which
    plane is visible. However, the framework imposes no limits on the sorts of
    objects that can be visualized; other examples include the
    \texttt{TextDisplay}, which shows strings, and the \texttt{TableDisplay},
    which shows tabular data as a spreadsheet. These plugins are typically used
    to display a module's typed outputs (i.e., its results).
  \item \textbf{\texttt{Tool}} -- A collection of rules binding user input
    (e.g., keyboard and mouse events) to display and data manipulation actions.
    For example, ImageJ2's \texttt{PanTool} pans a display when the mouse is
    dragged or arrow key is pressed; the \texttt{PencilTool} draws hard lines
    on the data within an image display. Many user interfaces render them as
    icons in the application toolbar.
  \item \textbf{\texttt{ConsoleArgument}} -- A plugin that handles arguments
    passed to the application as command line parameters. This plugin type
    makes the application's command line parameter handling extensible---a
    feature especially important for headless operation sans user interface.
\end{itemize}

This encapsulation of functionality, coupled with a plugin prioritization
mechanic, allows SciJava-based software to be fully customized or extended at
any point. An application such as ImageJ is then simply a collection of plugins
and services built on top of the SciJava Common framework. For instance, the
ImageJ Common \cite{imagej_common} component introduces new services
specifically for opening and displaying images, specializing the functions
defined in the lower-level components. Assigning these specialized functions a
higher plugin priority creates a natural, flexible ordering of operations.
Given that everything from user interfaces to file formats uses the SciJava
plugin mechanism, the path for overriding any behavior is clear and consistent.

Finally, to keep the plugin development process as simple as possible, great
care is taken throughout the codebase to adhere to interface-driven design with
default method implementations whenever possible. This strategy minimizes the
amount of code developers are responsible for writing, lowering the barrier to
entry for creating and modifying plugins.

\subsubsection*{Module framework}
To successfully interoperate with other scientific software, ImageJ algorithms
must be decoupled from the various user interfaces and applications which might
want to expose them to end users.

The key concept SciJava employs is that of \textbf{\textit{parameterized
modules}}: executable routines with declared input and output parameters of
specific types. These modules can take the form of \texttt{Command} plugins or
be expressed as scripts written in any supported scripting language (via
available \texttt{ScriptLanguage} plugins; see ``Plugin framework'' above). For
example, a user might write the following parameterized Groovy script:

\begin{quote}
  \small
  \begin{verbatim}
  // @INPUT String name
  // @INPUT int age
  // @OUTPUT String greeting
  greeting = "Hello, " + name + ". You are " + age + " years old."\end{verbatim}
\end{quote}

This script accepts two parameters as input---a name and an age---and outputs a
greeting based on the input values. Note the typing: the name can be any string
of characters, but the age must be an integer value; the greeting is also a
string of characters. Note also that this script makes no assumptions about
user interface; it is the responsibility of the framework to: A) prompt the
user for the input values in the most appropriate way, B) execute the module
code itself, and finally, C) process and/or display the output values in the
most appropriate way.

As such, this scheme has great potential for reuse across a wide
variety of contexts. For example, when running the above script from
the ImageJ user interface, a Swing dialog box will pop up allowing the
user to enter the name and age values; and after OK is pressed, the
greeting will be displayed in a new window. However, when running the
script headless from the command line interface, the input values can be
passed as command line arguments and the output values echoed to the
standard output stream. See Supplemental Figure 1 for an
illustration. Since many computational tools have this concept of
parameterized modules, SciJava developers need only create some general
adapter code to integrate the SciJava module framework with a given
tool---and then all SciJava modules become automatically available
within that tool's paradigm. We have already implemented such
integration for several other tools in the SciJava ecosystem, including
CellProfiler, \acrshort{knime} \cite{knip}, and the \acrshort{omero} image
server \cite{omero}.

SciJava Common has an important mechanism which facilitates the extensible and
configurable execution of modules: module pre- and post-processing. Developers
can write \texttt{ModulePreprocessor} and \texttt{ModulePostprocessor} plugins
to extend what happens when a module is executed (see ``Plugin framework''
above). Moreover, there are also two plugin types built on this module
processing mechanism which make it easy to customize and extend how modules
behave:

\begin{enumerate}
  \item {The process of collecting module inputs is known as \textit{input
    harvesting}. The \texttt{InputWidget} plugin type lets developers create
    widgets to harvest specific types of inputs from the user. In particular,
    the SciJava project provides Swing widgets for several data types
    (Supplemental Table 1).

    Some inputs are also automatically populated via
    \texttt{ModulePreprocessor} code. For example, when a single image
    parameter is declared, an ``active image preprocessor'' detects the
    situation, populating the value with the currently active image. In this
    way, the user does not have to explicitly select an image upon which to
    operate in the common case, but the module still has semantic knowledge
    that an image is one of the routine's input parameters.}
  \item The process of dealing with outputs after a module executes is known as
    \textit{displaying}. The \texttt{Display} plugin type lets developers
    visualize specific types of outputs in appropriate ways. The SciJava layer
    provides a basic display plugin for text outputs, which shows the text in a
    dedicated window, while the ImageJ layer provides additional similar
    display plugins for image and tabular data.
\end{enumerate}

One final SciJava subsystem of note is the \textit{conversion framework}, which
provides a general way of transforming data from one type to another. The
\texttt{Converter} plugin type lets developers extend SciJava's conversion
capabilities, allowing objects of one type to be used as module inputs of a
different type, in cases where the two types are conceptually analogous. For
example, data stored in memory as a \acrfull{matlab} matrix can be expressed as
an ImageJ image object, even though \acrshort{matlab} matrices are not natively
ImageJ images \cite{imagej_matlab}. When a suitable converter plugin is
present, modules capable of operating only on \acrshort{matlab} matrices become
transparently capable of accepting ImageJ images as inputs, thanks to the
framework's auto-conversion. Similarly, a converter between ImageJ and the
\acrfull{itk} \cite{itk} images greatly streamlines use of \acrshort{itk}-based
algorithms within ImageJ \cite{imagej_itk}.

\subsection*{ImageJ Common}

Meeting the needs of contemporary scientific image analysis requires a flexible
and extensible data model, including support for arbitrary dimensions, data
types and image sizes. To this end, we have chosen to model ImageJ2 images
using the ImgLib2 library, which itself provides an extensible,
interface-driven design that supports numeric (8-bit unsigned integer, 32-bit
floating point, etc.) and non-numeric data types. It also provides great
flexibility regarding the source and structure of data. Out of the box, ImgLib2
provides several data sources and sample organizations, including use of a
single primitive array (``array image''), one array per plane (``planar
image''), and block-based ``cell image.'' However, the library remains general
enough that alternative structures are also feasible. To quote the ImgLib2
article \cite{imglib2}:

\begin{quote}
  The core paradigm [of ImgLib2] is a clean separation of pixel algebra (how
  sample values are manipulated), data access (how sample coordinates are
  traversed), and data representation (how the samples are stored, laid out in
  memory, or paged to disc). ImgLib2 relies on virtual access to both sample
  values and coordinates, facilitating parallelizability and extensibility.
\end{quote}

ImageJ Common provides a unification of the type and storage-independence of
ImgLib2 with the SciJava Common plugin framework (described above). A
\texttt{Dataset} interface provides the fundamental representation of ImageJ
images, collections of images, and corresponding metadata: \acrfullpl{roi},
visualization settings, sample coordinates and physical calibrations, and much
more. Also provided are plugins and services for working with these
\texttt{Dataset} objects. Together, these classes form the access points for
higher-level components to open, save, generate and process these images.

Note that as of this writing, elements of the ImageJ Common data model and
corresponding services are still stabilizing. As such, we do not describe these
structures in technical detail here.

\subsection*{\acrshort{scifio}} An essential goal of ImageJ2 is to establish
universal image analysis routines, with no limits on application; however, the
proliferation of proprietary image formats from scientific instruments creates
a major obstacle to this ambition. To overcome this issue, the
\acrshort{scifio} core library establishes a common framework for reading,
writing and translating image data to and from the ImageJ Common data model, as
well as between domain-specific standard metadata models. \acrshort{scifio}
builds on the services provided in SciJava Common and ImageJ Common, defining
image \texttt{Format} and metadata \texttt{Translator} plugin types to
encapsulate the operations necessary to take an image source and standardize it
as an ImageJ \texttt{Dataset}.

\acrshort{scifio} builds upon SciJava Common's core I/O infrastructure, which
allows it to operate on most data locations independent of their nature.
SciJava Common provides a \texttt{Location} interface which acts as a data
descriptor, similar to a \acrfull{uri}. This \texttt{Location} interface is
specialized according to the nature of the data; for example, a
\texttt{URLLocation} identifies data served by a remote \acrshort{url}, while
an \texttt{OMEROLocation} (part of the ImageJ-\acrshort{omero} integration
\cite{imagej_omero}) identifies an image from an \acrshort{omero} server. For
data locations whose raw bytes can be accessed randomly and/or sequentially
(e.g., remote \acrshortpl{url}, but not \acrshort{omero} images), SciJava
Common provides a \texttt{DataHandle} plugin type which enables such access.
The core library provides \texttt{DataHandle} plugins for several kinds of data
locations, including files on disk, remote \acrshortpl{url}, and arrays of
bytes in local computer memory. Developers can easily create new
\texttt{DataHandle} plugins which provide random access into additional sorts
of locations, and \acrshort{scifio} will be able to use them transparently
without any changes to existing \texttt{Format} or \texttt{Translator} plugins.

The \texttt{Format} plugin \acrshort{api} is architected to support reading and
writing of image data in chunks, which provides scalability. It is no longer
necessary to have a large quantity of computer \acrshort{ram} to work with
large images---\acrshort{scifio} reads the data from the source location on
demand, paging it into and out of memory as needed. \acrshort{scifio}'s caching
mechanism persists any changes made to image pixels, even when chunks leave
memory, by using temporary storage on disk.

\acrshort{scifio} \texttt{Translator} plugins provide the means to translate
not only between image formats, but between common metadata models of various
scientific disciplines; for example, the
\acrshort{scifio}-\acrshort{ome}-\acrshort{xml} component provides a suite of
\acrshort{scifio} translators for converting between ImageJ Common data
structures and \acrshort{ome}-\acrshort{xml}, the data model of the
\acrfull{ome} \cite{ome_xml}. In this way, \acrshort{scifio} has the potential
to bridge interoperability gaps across various discipline-specific scientific
software packages.

Further details about \acrshort{scifio} can be found in the \acrshort{bmc}
BioInformatics software article ``\acrshort{scifio}: an extensible framework to
support scientific image formats'' \cite{scifio}.

\subsection*{ImageJ Ops}

ImageJ's ultimate purpose is image processing and analysis. To that end, we
have crafted the ImageJ Ops component: ImageJ2's shared, extensible library of
reusable image processing operations. As of version 0.33.0, the core Ops
library provides 788 \texttt{Op} plugins across nearly 350 types of ops in more
than 20 namespaces, covering functionality such as: image arithmetic,
trigonometry, Fourier transformations, deconvolution, global and local
thresholding, image statistics, image filtering, binary morphological
operations, type conversion, image transformations (scaling, rotation,
etc.)---even 2D and 3D geometric operations such as marching cubes 3D mesh
generation (see Figure 1 for examples).

\begin{quote}
[Figure 1: Examples of image processing algorithms available in ImageJ Ops.]
\end{quote}

ImageJ Ops was conceived with three major design goals: 1) easy to use and
extend; 2) powerful and general; and 3) high performance. To achieve all three
of these goals, Ops utilizes a plugin-based design enabling ``extensible case
logic.'' Ops defines a new plugin type, \texttt{Op}, each of which has a name
and a list of typed parameters, analogous to a function definition in most
programming languages. When invoking an op, callers typically do not specify
the exact \texttt{Op} plugin to use, but instead specify the operation's name
and arguments; the Ops framework then performs a \textit{matching} process,
finding the optimal fit for the given request. For example, calling
\texttt{math.add} with a planar image and a 64-bit floating point number leads
to a match of \texttt{net.imagej.ops.math.ConstantToPlanarImage.AddDouble},
which adds a constant value to each element of an image, whereas calling
\texttt{math.add} with two planar images results in a match of
\texttt{net.imagej.ops.math.IIToIIOutputII.Add}, which adds two images
element-wise.

This scheme is similar to---but much more powerful than---the method
overloading capabilities of many programming languages: op behavior can be
further specialized by tailoring \texttt{Op} implementations for specific
subclasses, generic parameters, and \texttt{Converter} substitutions (see
``SciJava Common'' above). Consider an op sqrt(image), which computes the
element-wise square root of an image. If we implement this op as
\texttt{sqrt(Dataset)}, we miss out on performance optimizations for
\texttt{ArrayImg}, an ImgLib2 container type where the entire collection of
image samples is stored in a single Java primitive array. However, if we only
implement \texttt{sqrt(ArrayImg)}, we are restricted in supported data types,
since not all images can be stored in such a manner. The power of the Ops
matching approach is that both of these and more can coexist simultaneously and
extensibly, and the most specific will always be selected at runtime.

The \texttt{Op} plugin type extends SciJava's \texttt{Command}, and therefore
all ops are SciJava parameterized modules, usable anywhere SciJava modules are
supported---see the ``module framework'' section in ``SciJava Common'' above.
Like standard modules, an op declares typed inputs and outputs. However, unlike
modules in general, an op must be a ``pure function'' with a fixed number of
parameters and no side effects; i.e., it must be deterministic in its behavior,
operating only on the given inputs, and populating or mutating only the given
outputs. These restrictions provide some very useful guarantees which allow the
system to reason about an op's use and behavior; e.g., after computing an op
with particular arguments once, the result can be cached to dramatically
improve subsequent time performance at the potential expense of additional
space. Properly constructed ops will also always be usable headless because
they do not rely on the existence of \acrshort{ui} elements.

\paragraph*{Op chaining and special ops}
It is often the case in image processing that an algorithm can be expressed as
a composition of lower level algorithms. For example, a simple difference of
Gaussians (``DoG'') operation is merely two Gaussian blur operations along with
a subtraction:

\begin{quote}
  $dog(image, \sigma_1, \sigma_2) =
  sub(gauss(image, \sigma_1), gauss(image, \sigma_2))$
\end{quote}

For users calling into the Ops framework via scripting, the core library
provides an \texttt{eval} op backed by SciJava's expression parser library,
which enables executing such sequences of ops via standard mathematical
expressions, including use of unary and infix binary operators.

For developers, the Ops library provides a mechanism for efficient
\textit{chaining} of ops calls. An op may declare other ops as inputs,
resulting in a tree of ops which are resolved when an op is matched; the
matched op instance can then be reused across any number of input values. In
this way, very general operations can be created to address a broad range of
use cases---e.g., the \texttt{map} operation provides a unified way of
executing an op such as \texttt{math.sqrt(number)} element-wise on a collection
(e.g., an image) whose elements are numbers. Indeed, in the case of DoG, the
Ops library's baseline implementation takes an image as input, along with two
\texttt{filter.gauss} ops and a \texttt{math.sub} op, and then invokes them on
the input image. The baseline \texttt{stats.mean} implementation is similar,
built on the \texttt{stats.sum}, \texttt{stats.size} and \texttt{math.div} ops.
Higher level DoG ops provide sensible defaults, enabling calls like
\texttt{dog(image, sigma1, sigma2)} to work, making common operations simple,
while leaving the door open for ultimate customization as needed.

To facilitate type-safe and efficient chaining of ops, the Ops library has a
subsystem known as \textit{special ops}. Such special ops are specifically
intended to be called repeatedly from other ops, without needing to invoke the
op matching algorithm every time. This repeat usage is achieved in a type-safe
and efficient way by explicitly declaring the types of the op's primary
inputs---i.e., the inputs whose values can be efficiently varied across
invocations of the op---as well as the type of the op's primary output.

Special ops have two major characteristics beyond regular ops. First, each
special op has a declared \textit{arity}, indicating the number of primary
inputs, which are explicitly typed via Java generics and can thus efficiently
vary across invocations of the op. Three arities are currently implemented:
\textit{nullary} for no inputs, \textit{unary} for one input, and
\textit{binary} for two inputs. It is important to note that unlike a formal
mathematical function, a unary special op may have more than one input
parameter---the ``unary'' in this case refers to the number of explicitly typed
parameters intended to vary when calling an instance of the op multiple times.
For instance, in the DoG example above, the baseline DoG is declared as a unary
op, so that the input image can vary efficiently while the sigmas etc. are held
constant.

Secondly, every special op is one of three kinds:

\begin{itemize}
  \item A \textit{function} operates on inputs, producing outputs, in a way
    consistent with the functional programming paradigm. Inputs are immutable,
    and outputs are generated during computation and subsequently also
    immutable. Functions are very useful for parallel processing since they are
    fully thread-safe even when object references overlap---but this safety
    comes at the expense of space, since they offer no way to reuse
    preallocated output buffers.
  \item A \textit{computer} is similar to a function, but populates a
    preallocated output object instead of generating a new object every time.
    Computers have many of the same advantages of functions, but provide the
    ability to reuse preallocated output buffers to improve efficiency in space
    and time.
  \item An \textit{inplace} op mutates its input(s) in place---i.e., its input
    and output are the same object. Inplace ops are highly space efficient, but
    lack the mathematical guarantees of functions and computers, since they
    destroy the original input data.
\end{itemize}

Some ops are implemented as \textit{hybrids}, offering a choice between two or
more of the function, computer and inplace computation mechanisms. Users of the
ops library---even advanced users---will rarely if ever need to know about this
implementation detail, but for developers crafting new ops, it is convenient to
have unifying interfaces which provide common logic for combining these
paradigms. See Supplemental Table 2 for a complete breakdown of the special op
kinds and arities.

\subsection*{ImageJ Legacy}
To maximize backwards compatibility with ImageJ 1.x, ImageJ2 must continue to
provide access to the complete existing \acrshort{ui} and \acrshort{api} with
which ImageJ users are familiar, while also making all new ImageJ2 features
available for exploration and use. Furthermore, to bridge the gap, ImageJ2 must
provide improved functionality transparently when possible, as well as support
seamless ``mixing and matching'' of the two respective \acrshortpl{api}. In
this way, ImageJ2 can enable gradual migration to the more powerful
capabilities of ImageJ2, while empowering developers' contributions to the
framework to be immediately effective. To achieve this goal, we identified the
major functional pathways of ImageJ 1.x and reworked them to delegate first to
ImageJ2 equivalents, falling back on the old behavior if needed.

There are two ImageJ components dedicated to maintaining backwards
compatibility with ImageJ 1.x. The lower level of the two is the IJ1-patcher:
using a tool called Javassist \cite{javassist} to perform an advanced Java
technique known as bytecode manipulation, ImageJ 1.x code is modified at
runtime to expose callback hooks at critical locations---e.g.: when opening
images with \textit{File $\triangleright$ Open\ldots}, closing the ImageJ
application, or displaying \acrshort{ui} components. These hooks are built
using the SciJava plugin infrastructure, allowing new behavior to be injected
into ImageJ 1.x despite the fact that it was not designed to support such
extensibility. In essence, ImageJ2 ``rewrites'' portions of ImageJ 1.x at
runtime to make integration possible. This approach is necessary because
altering ImageJ 1.x directly to enable such hooks would break backwards
compatibility with existing macros and plugins, ruining established scientific
workflows which have otherwise remained functional across many years.

By default, these hooks are exploited to inject ImageJ2 functionality in the
second compatibility layer: ImageJ Legacy. ImageJ2 intercepts an ImageJ 1.x
request and attempts to delegate to its own routines. For example, in our
implementation of the \textit{File $\triangleright$ Open\ldots} hook, we use
the SciJava I/O service, which provides extensible support for data types via
SciJava I/O plugins. This allows the full power of \acrshort{scifio} to be
called automatically by \textit{File $\triangleright$ Open\ldots} without
requiring users to select individual loader plugins. In this way, ImageJ2
exposes new ``seams'' which provide extensibility points not available in the
standalone ImageJ 1.x project \cite{legacy_code}.

A second major function of ImageJ Legacy is to provide a wrapping legacy
\acrshort{ui}: an ImageJ2 \texttt{UserInterface} plugin that reuses the ImageJ
1.x \acrshort{ui}, but maintains synchronization between respective data
structures. For example, consider the \texttt{ImagePlus} structure in ImageJ
1.x and its equivalent, the \texttt{Dataset}, in ImageJ2. By default, an
\texttt{ImagePlus} and \texttt{Dataset} could not be interchanged; they have
different Java class hierarchies, and with ImageJ2's expanded data model, a
\texttt{Dataset} is much more expressive than an \texttt{ImagePlus}. However,
requiring plugins to ``select one'' would impose a technical barrier, even if
both structures are available in the same application. Thus, the legacy
\acrshort{ui} notes when either an \texttt{ImagePlus} or a \texttt{Dataset} is
created and ensures a complementary instance is mapped, via SciJava
\texttt{Converter} plugins. This brings the ImageJ 1.x and ImageJ2 worlds
closer together: when an image is opened, it can be used by plugins that would
take an \texttt{ImagePlus} or a \texttt{Dataset} regardless of whether that
image was opened via an ImageJ 1.x or ImageJ2 mechanism. Furthermore, because
conversion is handled in the ImageJ Legacy layer, individual plugins do not
require knowledge of the synchronization.

Shared image data structures are but one aspect of the legacy \acrshort{ui}'s
synchronization. Others include logging, notification, and status
events---essentially all \acrshort{ui} events are mapped across paradigms.
Whenever possible, these conversions are achieved using an adapter class that
implements a common interface (e.g., \texttt{Dataset}), which wraps the object
of interest (e.g., \texttt{ImagePlus}) by reference. This approach enables
information to be translated between ImageJ 1.x and ImageJ2 structures on
demand, while minimizing the performance impact. Wrapping by reference also
mitigates the burden of updates; once synchronization is established, changes
to the underlying object are automatically reflected in the wrapper.

\subsection*{ImageJ Updater}
The ImageJ Updater is the mechanism by which the available and installed
components of ImageJ are managed. At its core, the Updater is a flexible
component for tracking ImageJ update sites: endpoints containing versioned
collections of files. Users can pick and choose which update sites they wish to
enable, with ImageJ's core functionality offered on the base ``ImageJ'' update
site, which is on by default. Distributions of ImageJ such as \acrfull{fiji}
\cite{fiji} extend this base with additional functionality (Figure 2) in the
form of more plugins, scripts, sample images, \acrlongpl{lut}, etc., leveraging
the ability to override ImageJ's base behavior using SciJava's plugin priority
mechanism (see ``SciJava Common'' above for details).

\begin{quote}
[Figure 2: ImageJ update sites provide additional functionality to ImageJ.]
\end{quote}

The Updater stores metadata in a \texttt{db.xml.gz} file in the root of each
update site, which describes the files that are part of that update site,
including checksums and timestamps for all previous versions. In this way, the
Updater can tell whether each local file is: A) an up-to-date tracked file; B)
an old version of a tracked file; C) a locally modified version of a tracked
file; or D) an untracked file. Update sites are served to users over
\acrshort{http}. Developers may upload files to an update site via an
extensible set of protocols, as defined by Uploader plugins. The core ImageJ
distribution includes two such plugins---one for
\acrshort{ssh}/\acrshort{scp}/\acrshort{sftp} and one for
\acrshort{webdav}---but in principle, the Updater makes no assumptions about
how files are uploaded.

The \texttt{db.xml.gz} structure was originally designed for use with the
\acrshort{fiji} Updater, the ImageJ Updater's predecessor. The logic of the
\acrshort{fiji} Updater was migrated into the core of ImageJ2, with backwards
compatibility preserved for existing \acrshort{fiji} installations. As part of
that migration, the Updater was heavily refactored to be \acrshort{ui}
agnostic, such that additional user interface plugins for the Updater could be
created which leverage the same core. Out of the box, ImageJ provides two
different user interfaces for the Updater: a command-line tool intended for
power users and developers, and a Swing \acrshort{ui} intended to make updating
easy for end users. When ImageJ is first launched, it automatically runs the
``Up-to-date check'' command, which then displays the Updater \acrshort{ui} if
updates are available from any of the currently enabled update sites.

%% The user interface should be described and a discussion of the intended uses
%% of the software, and the benefits that are envisioned, should be included,
%% together with data on how its performance and functionality compare with,
%% and improve, on functionally similar existing software. A case study of the
%% use of the software may be presented. The planned future development of new
%% features, if any, should be mentioned.

\section*{Results and Discussion}
ImageJ has transformed from a single-user, single-bench application to a
versatile framework of extensible, reusable operations. In the following
sections, we discuss how each core aspect of ImageJ2 has impacted community
usage and how we expect these qualities to shape future developments.

\subsection*{Functionality}

The architecture of ImageJ2 enables it to meet current and future demands in
image analysis.

\textbf{Dimensions.} Using ImgLib2 opens up caching options for operating on
extremely large images, an area in which ImageJ 1.x has previously struggled.
ImageJ 1.x is inherently limited to five dimensions (X, Y, Z, time, and
channel) with fewer than $2^{31}$ pixels per XY plane, e.g. a $50,000 \times
50,000$ plane being too large to represent. ImageJ 1.x allows composite images,
but is constrained to a maximum of seven composited channels. ImageJ2's
N-dimensional data model supports up to $2^{31} - 1$ dimensions, each with up
to $2^{63} - 1$ elements, and composite rendering over any dimension of
interest regardless of length. There are several preset dimensional axis types,
and new types can also be defined as needed. When visualizing multi-channel
data, each channel can now have its own \acrfull{lut} without constraint.

\begin{quote}
[Figure 3: ImageJ 1.x case logic compared to a unified ImgLib2 implementation.]
\end{quote}

\textbf{Types.} ImageJ 1.x supports only five image types for representing
sample values: 8-bit unsigned integer grayscale, 8-bit with a color lookup
table, 16-bit unsigned integer grayscale, 32-bit floating point, and a 32-bit
integer-packed color type representing three 8-bit unsigned color channels:
red, green, and blue. Furthermore, this support is highly static, sometimes
requiring case logic for algorithms to properly handle each of desired image
types independently. In contrast, ImgLib2 is explicitly designed to facilitate
algorithms developed agnostic of image type (Figure 3). ImageJ2 already
supports over twenty different image types (Supplemental Table 3), including
arbitrary precision integer and decimal types, and further types are definable
using SciJava Common's flexible plugin framework. SciJava \texttt{Converter}
plugins also extend the reach of ImageJ2-based algorithms even further into
additional data structures, such as \acrshort{matlab} matrices
\cite{imagej_matlab} and \acrshort{itk} images \cite{imagej_itk}.

\textbf{Storage.} The prime example of an alternate storage source in ImageJ
1.x is the virtual stack, allowing image slices to be read on demand---e.g., if
the image would not normally fit in memory. However, ImageJ 1.x commands must
explicitly account for whether or not they can operate on a virtual stack,
requiring a proliferation of case logic and complexity. ImageJ2 takes advantage
of ImgLib2's extensible container system, which enables data to be stored
flexibly: as files on disk, remote \acrshortpl{url}, within a database,
generated on-the-fly, etc. Such routines can even be used with pixel and
storage types implemented well after their creation without having to change
the original implementation. As image acquisition sizes increase, we expect
virtualized image data to be particularly critical to the future of image
analysis. The \acrshort{scifio} library already provides an ImgLib2 image type
(``\acrshort{scifio} cell image'') that supports block-based read/write caching
from disk, effectively behaving as a writeable virtual stack.

\textbf{\Acrfullpl{roi}.} Like ImageJ 1.x, ImageJ2 provides support for
\textit{\acrfullpl{roi}}, which are functions that identify samples upon which
to operate, as well as \textit{overlays}, which are visuals (e.g., text)
superimposed for visualization. ImageJ2 builds upon the \acrshort{roi}
interfaces of ImgLib2, allowing for any number of simultaneous \acrshortpl{roi}
and overlays to be associated with a particular image without the need for
additional tools like ImageJ 1.x's global \acrshort{roi} Manager window.

Because \acrshortpl{roi} are part of the core ImgLib2 library, it is possible
to process subsets of images identified by one or more \acrshortpl{roi} using
an ImgLib2-based algorithm, and the Ops library can process data within a
\acrshort{roi} as a single functional operation. This continues ImageJ2's
migration towards image processing algorithms that need not add explicit case
logic---e.g., to handle \acrshortpl{roi} separately---but instead simply
provide a pixelwise function, or iterate using ImgLib2's generic iteration
mechanism. In this way, we continue to reduce the effort and complexity of
ImageJ2 plugins, while increasing their utility and application.

\textbf{Modularity.} ImageJ 1.x was developed with a ``single computer, single
user, single operation'' in mind. Although ImageJ 1.x can be used as a library,
it will always be a single unit that cannot be decoupled from its dependencies,
which are implicit in its source code. ImageJ2 has succeeded in building a
cohesive application from encapsulated, modular components unified by the
SciJava plugin framework. Each component is independently deployed and
accessible via the build automation tool Maven \cite{apache_maven}, allowing
developers to pick and choose the individual pieces relevant to them---be it
the ImageJ application, a particular scripting language, image format, or the
SciJava Common core. SciJava-based projects inherit a ``bill of materials''
which enables components to be combined at versions known to be compatible with
each other \cite{imagej_architecture}. We have already seen the benefits of
this modularity---for example, the use of the \acrshort{scifio} library in
\acrfull{knip} to produce images compatible with ImageJ2 commands.

\textbf{Ops.} The ImageJ Ops library is the centerpiece of ImageJ2, bringing
Java's mantra of ``write once, run anywhere'' to image processing algorithms.
Ops provides a wealth of image processing algorithms to users, accessible in a
unified way that empowers developers to transparently extend and enhance the
behavior and capabilities of each operation. It is critical to appreciate that
each type of op (more than 350 different operations as of version 0.33.0)
represents a potential extension point for optimizing existing parameters, or
supporting new ones. In contrast to algorithms coded using ImageJ 1.x data
structures, all ops work without modification on all image types (Supplemental
Table 3) and containers, including those not yet in existence. As the Ops
project is a very active collaboration across several institutions including
the University of Konstanz, University of Wisconsin-Madison and others, we
expect the core library to continue to grow in both available functionality and
use within the community.

\subsection*{Extensibility}

\textbf{Plugins.} The ImageJ2 plugin framework, built on top of SciJava Common,
is a modular and extensible infrastructure for adding features. Plugins can now
take many forms, including image processing operations, new tools, and even
completely new displays. In ImageJ 1.x there are three kinds of plugins: 1) the
standard \texttt{PlugIn}, which provides a freeform \texttt{run(String arg)}
method; 2) \texttt{PlugInFilter}, which processes images one plane at a time;
and 3) \texttt{PlugInTool}, which adds a function to the toolbar. In ImageJ2,
these ideas are expressed in the form of \texttt{Command}, \texttt{Op} and
\texttt{Tool} plugins, respectively---although these plugin types have many
advantages over their ImageJ 1.x analogues: type-safe chaining of operations,
dynamic selection of ops based on arguments, \acrshort{ui} agnosticism, etc.
Furthermore, many other types of plugins are available as well, and the
flexibility of the SciJava plugin framework also allows for additional new
types of plugins to be defined.

\textbf{Modules.} The ImageJ application's menu structure is made up of SciJava
modules---most commonly commands and scripts. Thus, scripts and
\texttt{Command} plugins are probably the most common points of extension for
developers exploring the ImageJ2 architecture. Writing such extensions in
ImageJ2 is much simpler than in ImageJ 1.x, which requires each extension to
explicitly create its own dialog box to collect user input. In ImageJ2, the use
of parameters results in more declarative extensions, freeing software
developers from the need to explicitly ask the user for input values in the
vast majority of cases, and substantially reducing boilerplate and
\acrshort{ui}-specific code, making commands shorter and easier to understand
(see Figure 5 in ``Usability'' below). Moreover, this mechanism makes ImageJ2
modules truly independent of the user interface, allowing them to work with any
\acrshort{ui} or headlessly. The module simply declares its inputs and outputs
using the appropriate parameter syntax, and lets ImageJ do the rest.

\textbf{Formats.} In an open source image analysis program like ImageJ, an
extensive collection of supported image formats is necessary to maximize
relevance and impact across the community. ImageJ 1.x provided a central
\texttt{HandleExtraFileTypes} class to enable extensibility, but required
direct modification of this class to do so, resulting in many third parties
each shipping their own modified version. Only one modification could ``win,''
effectively breaking any other supported formats. To fill this role in ImageJ2,
the \acrfull{scifio} library provides extensible image format support tailored
to the ImageJ Common data model.

As of version 0.29.0, the core \acrshort{scifio} library provides a collection
of \textasciitilde{}30 open formats, and also includes a wrapping of the
Bio-Formats library \cite{bio_formats}, which enables a wide variety of
supported images throughout all ImageJ operations. Furthermore,
\acrshort{scifio} enables developers to create their own \texttt{Format}
plugins to smoothly integrate support for new proprietary formats and metadata
standards without modification of core functions or proliferation of one-off
format commands.

\textbf{Image processing.} ImageJ's ultimate goal is to be a supremely powerful
tool for image processing; it is therefore paramount that ImageJ include a
world-class extension mechanism for image processing algorithms. ImageJ2's op
matching subsystem offers extensible case logic: an \texttt{Op} plugin can be
written to add a new algorithm, to extend an existing algorithm to support new
data structures, or to make an algorithm more efficient for specific data
types, all without impacting previously written code. As the Ops library
matures, we expect to see new \texttt{Op} implementations along all of these
lines in existing third-party suites, conveniently shipped to users via ImageJ
update sites. Hence, unlike in ImageJ 1.x, existing user scripts using the Ops
library will automatically benefit from new performance-enhancing ops.

\textbf{User interface.} ImageJ 1.x's user interface is written in Java
\acrshort{awt} with many assumptions throughout the codebase relying on this
fact. Hence, ImageJ 1.x is only limitedly usable in a headless way (e.g., for
image processing on a server cluster). Normally, ImageJ 1.x cannot be used
headless at all: some lynchpin ImageJ 1.x classes---notably \texttt{ij.ImageJ}
and \texttt{ij.gui.GenericDialog}---derive from \texttt{java.awt.Window}, and
such classes cannot be instantiated when running in headless mode. Fortunately,
the ImageJ Legacy layer's runtime patcher rewrites affected ImageJ 1.x classes
to derive from non-\acrshort{awt} window classes when in headless mode; as
such, ImageJ2 makes headless execution of ImageJ 1.x scripts feasible.

Furthermore, ImageJ 1.x's reliance on \acrshort{awt} also limits its ability to
be embedded into other applications using different \acrshort{ui} frameworks,
such as Swing or Eclipse \acrshort{swt}. While some applications have succeeded
in doing so \cite{bio7}, the amount of work required in response to each ImageJ
1.x update is considerable, since many changes must be made to the ImageJ 1.x
core source code.

In contrast, ImageJ2's separation between the underlying data model and the
user interface enables it to run headless or within a variety of different user
interface paradigms with no changes to the core. Developers can create their
own plugins to provide alternative user interfaces. ImageJ2 is even capable of
displaying multiple \acrshortpl{ui} simultaneously in the same Java runtime.
Making ImageJ support multiple user interfaces in this way is surely worth the
effort for many years to come; one of the most common questions about ImageJ
from software developers is how to customize the ImageJ \acrshort{ui}, for
which ImageJ2 now provides a clear mechanism. Adding support for a new
\acrshort{ui} now only requires writing a new user interface plugin and
corresponding display and widget plugins.

While the current flagship user interface for ImageJ2 is still the ImageJ 1.x
\acrshort{ui} via the ImageJ Legacy component, ImageJ2 also has a Swing user
interface modeled after the ImageJ 1.x \acrshort{ui}, but which stands alone
with no dependence on ImageJ 1.x code. We have been successful in
``reskinning'' this Swing \acrshort{ui} with various Java \acrfullpl{laf},
including the Metal, Motif, Nimbus, Aqua, Windows and \acrfull{gtk}
\acrshortpl{laf}. Furthermore, we explored proof-of-concept \acrshort{ui}
implementations in other frameworks, such as Eclipse's \acrfull{swt}, Java
\acrfull{awt} sans Swing, and Apache Pivot. There is also a JavaFX
\acrshort{ui} for ImageJ2 developed by Cyril Mongis at the University of
Heidelberg \cite{imagejfx}, as well as integrations of ImageJ into other
powerful end-user applications such as \acrshort{knime} and CellProfiler. See
Figure 4 for a side-by-side illustration of \acrshortpl{ui}.

\begin{quote}
[Figure 4: Side-by-side comparison of ImageJ2-based user interfaces and
integrations.]
\end{quote}

\textbf{Interoperability.} There is no one-size-fits-all tool for scientific
image processing. A diversity of tools benefits users, even more so when they
can interoperate. ImageJ 1.x was designed to be run by a single user on a
single desktop computer. Many aspects of the program are structured as
singletons: one macro interpreter, one \texttt{WindowManager}, one active
image, one \texttt{PlugInClassLoader}, one active \acrshort{roi} per image, one
set of overlays, one active \acrshort{roi} in the \acrshort{roi} manager, etc.
This structure imposes many limitations: for example, multiple macros cannot
run concurrently, and it is not possible to operate more than one instance of
ImageJ 1.x in the same \acrshort{jvm} simultaneously---e.g., on a single web
page as applets.

ImageJ2 is structured as an application container, avoiding the static
singleton pattern and hence many of ImageJ 1.x's limitations. Multiple
instances of ImageJ2 can run in the same \acrshort{jvm}, each with multiple (or
no) user interfaces and multiple concurrent operations. Our primary goal is to
make each encapsulated component of ImageJ2 usable in other software projects.

There are several examples of external developers leaning on this generality to
expose SciJava modules in interesting ways: for example, automated conversion
to nodes in a \acrshort{knime} workflow. Continuing efforts are underway to
foster further interoperability with several other projects, including
\acrshort{itk}, \acrshort{omero}, CellProfiler, \acrshort{matlab},
\acrshort{alida} and \acrshort{mitobo} \cite{mitobo}.

\subsection*{Reproducibility}

In the interest of transparency and reproducibility---especially in the context
of open science---the ImageJ2 project strives to be accessible. Ultimately, we
want to spur the community to improve ImageJ in a collaborative way, by
providing open access to resources. Of course, we recognize the need for
responsive, reliable maintainers to coordinate and facilitate contributions.
However, with the Internet's modern software infrastructure, it is now quite
feasible to push ImageJ development in a more collaborative and
community-driven direction, embracing the ``GitHub effect''
\cite{github_effect} of worldwide, distributed development.

All ImageJ2 source code is open and publicly available on GitHub
\cite{imagej_source_code}, and all core components are permissively licensed
\cite{imagej_licensing} to avoid any ambiguity over how the code can be used.
But visibility alone is not sufficient to keep a project open; each line of
code adds complexity, making the project harder to understand and maintain.
Modular, encapsulated design, the application of the ``\acrfull{dry}'' concept,
and avoidance of overly ``clever'' code keeps ImageJ2 well-organized and easier
to understand. Extensive online documentation \cite{imagej_web_site} and
Javadoc \cite{imagej_javadoc} provide further insight, while effective unit
testing and dedicated tutorial components illustrate concrete use cases. By
keeping a clean, well-organized and well-documented codebase, we facilitate
community contributions, as well as continued maintenance of ImageJ into the
future.

ImageJ2's open development process provides many benefits over the centralized
process of ImageJ 1.x. The use of Maven makes dependency management human
readable and enables the use of a ``bill of materials'' to unambiguously
determine which versions of each ImageJ component are compatible. The use of
Git has evolved revision control to a new level of documentation, clearly
communicating why changes are made and encouraging atomic, easily understood
changes. Furthermore, ImageJ2 minimizes the barrier to community contributions
via an open issue tracking system \cite{imagej_issues} and open patch
submission process \cite{imagej_contributing}. We also maintain a \acrfull{ci}
server \cite{imagej_jenkins} which runs automated regression tests whenever
modifications to ImageJ's source code are published; such tests help to avoid
and detect regression bugs so that core functionality and behavior will
continue to work reliably as the program evolves. This same CI server manages
release and deployment, javadoc generation, website validation, and more. On
the application side, the ImageJ Updater functions in a way similar to version
control, with all prior binaries retained and metadata in \acrshort{xml}
format.

\subsection*{Usability}

The ImageJ community includes both end users---who use ImageJ as an application
and want it to ``just work''---and software developers---who want to customize
and invoke parts of ImageJ as a software library from their own programs.
However, these roles are not rigid; many users write scripts and macros to
facilitate their image analysis, and many developers also use ImageJ as an
application. ImageJ2 includes a powerful Script Editor with many tools to aid
users as they transition into the realm of development. This tool removes much
of the complexity of traditional software development, allowing users to focus
on coding without the added burden of applying compilers, \acrshortpl{ide}, or
the command line.

In addition, the SciJava parameterized scripting mechanism makes it easier for
users to write scripts whose inputs and outputs are declared in a clear and
straightforward manner. SciJava parameters reduce the boilerplate code
historically needed to define a script's input values, in some cases by 50\% or
more (Figure 5). Leveraging SciJava annotations also frees the plugin from the
Java \acrshort{awt} dependencies of ImageJ 1.x's \texttt{GenericDialog},
allowing the plugin to be used headlessly, in future \acrshortpl{ui}, and even
in other applications.

\begin{quote}
[Figure 5: Comparison of pure ImageJ 1.x command with one using SciJava
declarative syntax.]
\end{quote}

\textbf{Sensible defaults.} A key example of reasonable default behavior is the
SciJava conversion framework with its specialized \texttt{Converter} plugin
type. \texttt{Converter} plugins define useful type substitutions that would
not normally be allowed by the Java type hierarchy. For example, conversion of
ImageJ \texttt{Dataset} objects to and from other paradigms (\acrshort{matlab}
arrays, \acrshort{itk} images, etc.) is facilitated by \texttt{Converter}
plugins which encapsulate the logic for each conversion case. The framework
then uses the converters automatically when modules are executed, so when a
user says e.g. ``run this \acrshort{itk} algorithm on this dataset I opened''
everything ``just works'' without the user needing to perform an explicit
manual conversion. From a software development perspective, this scheme lets
ImageJ2 retain the advantages of strong typing while escaping the corresponding
shackles that often accompany it.

The ImageJ Ops library provides another illustration of ImageJ2's sensible
defaults structured in layers. While every op in the system is a dynamically
callable plugin, the core Ops library also organizes its built-in operations
into a centrally accessible collection of type-safe namespaces in a standard
Java \acrshort{api} explorable from \acrshortpl{ide} like Eclipse e.g. via its
Content Assist functions. This structure also provides an elegant and
easy-to-read syntax for calling ops in script-driven workflows (see Figure 7 in
``Compatibility'' below).

\textbf{Automatic updates.} In ImageJ 1.x, plugin installation requires users
to download a \acrshort{jar} or Java class file and place it within the ImageJ
plugins folder. Updating an installed plugin essentially requires repeating the
manual installation steps, which is both tedious and error-prone. Developers
have to manually manage their plugin's dependencies, which in practice leads to
users receiving cryptic error messages when multiple plugins require
incompatible component versions. Even worse, some developers then make
suboptimal design decisions to work around this difficulty, such as
reimplementing functionality already provided by well-tested third party
libraries, and/or creating so-called ``uber-\acrshortpl{jar}'' which lump
together the dependencies into intractable bundles \cite{imagej_uber_jar}.

The ImageJ Updater vastly simplifies this process by informing users
automatically when a new plugin version is available and enabling single-click
upgrades to the latest version of all components. On the development side, the
use of Maven by the ImageJ2 and \acrshort{fiji} projects provides a clear best
practice for managing dependencies in a consistent way, which reduces the
chance of broken end-user installations.

ImageJ's support for multiple update sites makes it feasible for community
developers and third parties to create their own update sites from which users
can pick and choose, without them needing to become a part of the core ImageJ
or \acrshort{fiji} distribution. This distributed model of update sites fits in
very well with the community-driven aspects of ImageJ, dramatically lowering
the barrier for sharing effort. This capability is made even more potent by the
Personal Update Sites feature, which lets users link their ImageJ wiki account
to their own personal web space. Furthermore, the Updater derives its initial
list of available update sites from the ``List of update sites'' wiki page of
the ImageJ website \cite{imagej_list_of_update_sites}---plugin developers can
edit this wiki page in the same way as the rest of the ImageJ website, in order
to make their site automatically available to all users of ImageJ. Editing this
page is not mandatory, however; users can also manually edit their ImageJ
installation's list of available update sites---e.g., if their organization
offers an internally managed update site for plugins specific to their
institute.

Although manual plugin installation is still supported in ImageJ2, many
organizations have already publicized their own update sites, and thanks to the
Updater together with the ImageJ Legacy layer, all of the plugins served from
those sites are easily accessible within ImageJ2. This has helped to focus the
\acrshort{fiji} project on its original goal of being a curated collection of
plugins facilitating scientific image analysis. In addition to \acrshort{fiji},
hundreds of third-party update sites are available, such as \acrshort{loci},
\acrfull{bar}, BioVoxxel, the Stowers Institute, and the BioImaging and Optics
platform of the \acrfull{epfl}, most of which are served from the centrally
managed Personal Update Sites server \cite{imagej_sites}.

\subsubsection*{Performance}

ImageJ2 is engineered to accommodate the growing size and complexity of image
data. Although performance has been a serious design consideration, we believe
in aggressive performance optimization only on an as-needed basis as software
components stabilize and mature \cite{premature_optimization}. By designing a
robust framework that allows for specialization at every level, we avoid
compromising design for the sake of incremental performance gains, while
empowering developers to optimize when necessary. Furthermore, the fact that
ImageJ2 maintains 100\% backwards compatibility with ImageJ 1.x (see
``Compatibility'' below) means that existing high-performance image processing
approaches continue to work as is, even if they do not benefit from ImageJ2's
architectural improvements.

\textbf{Efficiency.} The time performance of ImageJ2 data structures is
generally consistent with those of ImageJ 1.x. The core of performance in
ImageJ2 hinges on the efficiency of the various ImgLib2 containers. We have run
benchmarks comparing the time performance of iteration and access on ImgLib2
image structures with that of ImageJ 1.x images, as well as compared to raw
array access \cite{imglib2_benchmarks}. We found that thanks to Java's
\acrfull{jit} compiler, ImgLib2 is highly comparable to ImageJ 1.x in these
regards (Supplemental Figure 2).

When time performance is dominated by the overhead of looping itself, some
ImgLib2 container types such as cell images may be measurably slower to iterate
and access. However, this loop overhead is generally very small, and for most
container types (e.g., array and planar images) the \acrshort{jit} quickly
optimizes the code to equal the speed of raw array access. Hence, for
non-trivial image processing operations which take significant time to compute
per sample, overall time performance converges across all data structures and
container types, ImageJ 1.x and ImageJ2 alike.

The advantages of ImgLib2's type- and container-agnostic algorithm development
outweigh any minor differences in time performance, saving developer time and
effort via simpler, more maintainable code. Furthermore, ImgLib2's more
comprehensive set of image types (Supplemental Table 3) make it easier to
optimize for space efficiency. For example, an image sequence recorded using a
12-bit detector requires 16 bits per sample in ImageJ 1.x, whereas ImageJ2 can
pack that data without wasted bits using ImgLib2's \texttt{uint12} data type,
resulting in a 33\% increase in space efficiency.

Relatedly, the design of the ImageJ Ops library realizes these same efficiency
advantages. End user scripts invoke ops by name and arguments, and the Ops
matching algorithm takes care of selecting the implementation optimized for
those arguments. This scheme enables image processing algorithms to be written
once, then automatically benefit from future performance optimizations without
explicit case logic.

\begin{quote}
[Figure 6: Benchmarks of a simple addition operation with ImageJ Ops and ImageJ
1.x.]
\end{quote}

To validate this approach, we benchmarked the core Ops library's math.add
operations which add a constant value to each element of an image (Figure 6).
As evidenced by the results, the inplace ImageJ 1.x version of this operation
(the \textit{Add\ldots} command under Math in the Process menu) performs much
better than some generalized op implementations (\texttt{II} source to
\texttt{RAI} destination) which work on all image types---but the optimized ops
outperform it, with the single-threaded inplace \texttt{ArrayImg} op finishing
7 times faster, and the multithreaded version finishing 9 times faster. While
some of this gain is likely due to the expense of ImageJ 1.x's bounds checking,
it is also evident from the results that the optimized ops are comparable in
efficiency to operations on raw primitive arrays.

\textbf{Scalability.} As discussed in ``Functionality'' above, ImageJ 1.x is
fundamentally limited to XY image planes of less than $2^{31}$ pixels due to
its use of one Java primitive array per plane, and to the size of available
computer memory for many of its image processing operations. In contrast,
ImageJ2 has been engineered at every level to support scalable image processing
using cell images which are cached to and from mass storage on demand.
ImageJ2's careful separation of concerns and enhanced command line parameter
handling enable ImageJ to run headless on remote servers, opening up a wide
array of possibilities for scalable computation. The \acrshort{scifio} library
enables direct access into image data samples, so that image data many orders
of magnitude larger than available computer memory can be systematically
processed on an individual workstation or using a cluster. And thanks to
visualization tools like the BigDataViewer plugin \cite{bigdataviewer}, which
is also built on ImgLib2 cell images, it is now realistic to quickly visualize
and explore such massive datasets.

\subsection*{Compatibility}
As ImageJ2 continues to mature, ImageJ 1.x functionality will be increasingly
replaced with more powerful ImageJ2 equivalents: image processing algorithms
built on ImageJ Ops, data format plugins built on \acrshort{scifio}, etc. But
this process is both lengthy and necessarily incomplete: migrating ImageJ 1.x
core functionality alone is a years-long process as the ImageJ2
\acrshortpl{api} continue to evolve, mature and stabilize---and there are
countless other useful plugins and scripts in the wild, some of which will
never be updated to the new \acrshortpl{api}. Meanwhile, development of ImageJ
1.x also progresses, with users continuing to request bug fixes and new
features within its intended scope. As such, the importance of a robust
transitional strategy for migrating from ImageJ 1.x to ImageJ2 cannot be
overstated.

Fortunately, thanks to the ImageJ Legacy and IJ1-patcher components, the
community can blend usage of ImageJ 1.x and ImageJ2 functionality,
cherry-picking the best from each world to accomplish their image analysis
tasks. Parameterized ImageJ commands and scripts may continue to use ImageJ 1.x
data structures and plugins as needed, while also taking advantage of ImageJ2
functionality as appropriate (Figure 7), and declaring and populating input
values with less boilerplate code (see Figure 5 in ``Usability'' above).

\begin{quote}
[Figure 7: A mixed-world ImageJ 1.x + ImageJ2 script.]
\end{quote}

Although the development of ImageJ2 has necessitated re-implementation of
ImageJ 1.x functionality, maintaining backwards compatibility with ImageJ 1.x
will remain a fundamental goal. Abandoning or ignoring ImageJ 1.x would have
been a significant disservice to the community, causing a rift to the detriment
of all parties. Our efforts to enable incremental migration from ImageJ 1.x to
ImageJ2 allow the two projects to continue developing in tandem, with new
features in each reaching a unified ImageJ community.

\subsection*{Community}

Ultimately, the goal of ImageJ is to enable scientific collaboration and
achievement, which requires community management as much as code management.
ImageJ 1.x leverages open-source code, a public web site and a mailing list to
support discussion and contributions from people across the globe. However, it
follows a centralized ``cathedral'' development model, rather than a
community-driven ``bazaar'' style model \cite{cathedral_bazaar}, with its
primary resources and scalability fundamentally limited by a single
``gatekeeper.'' For ImageJ's continued success and growth, it is critical to
renew its focus on partnership and communication with the community
\cite{imagej_communication}.

\textbf{Online resources.} The centrally organized documentation of ImageJ2
takes the form of a collaborative wiki \cite{imagej_web_site} with over 800
articles: a ``world-writable'' location for both users and developers to learn
about and contribute to ImageJ. The wiki is complemented by the ImageJ Forum
\cite{imagej_forum}, a powerful, friendly and universally accessible discussion
channel driven by the excellent Discourse software, which is engineered to
encourage civil interaction \cite{discourse}. Finally, ImageJ's source code and
issue tracking via GitHub completes the community-centric approach for managing
and discussing changes and improvements.

These resources together enable project developers to clearly communicate the
expectations and norms surrounding plugin development, contribution,
maintenance and support, empowering users to easily see who is responsible for
each plugin as well as its support and development status \cite{imagej_team},
outstanding bugs \cite{imagej_issues} and future plans \cite{imagej_roadmap}.
This is a critical service for the community: it is not enough to provide a
convenient way for people to publish, share and consume extensions---we have
learned from experience that there must be a social framework in place for
managing and understanding the software development lifecycle of the myriad
community efforts.

\subsection*{Future directions}
ImageJ is more than a single application: it is a living ecosystem of
scientific exchange. As acquisition technology continues to advance, there will
always be a need for new development and maintenance within ImageJ. There are
still key technical tasks remaining for ImageJ2 to achieve stability, as well
as new directions made possible by the ImageJ2 platform which we are excited to
explore:

\begin{itemize}
  \item Finalize the ImageJ Common data model to support extensible attachment
    of metadata, including spatial metadata, that respond robustly to image
    processing operations such as transformation.
  \item Extend ImgLib2's N-dimensional \acrshort{roi} interfaces to cover all
    needed cases.
  \item Update the core SciJava I/O mechanism to be plugin-driven for improved
    extensibility of data source location.
  \item Generalize \acrshort{scifio}'s planar model to operate on arbitrary
    ``blocks'' at a fundamental level.
  \item Retire the custom C-based ImageJ desktop application launcher,
    migrating to the industry-standard application bundling of JavaFX.
  \item Complete our ongoing effort to automate the documentation regarding
    development and maintenance responsibility for every core component of the
    ImageJ ecosystem, including \acrshort{fiji} components \cite{imagej_team}.
  \item Develop our ImageJ-based \acrshort{rest} image server prototype toward
    production use, providing a common language- and implementation-independent
    \acrshort{api}.
  \item Implement a web \acrshort{ui} built on the \acrshort{rest} image
    server.
  \item Improve the ImageJ Updater user interface to be more user friendly, so
    that users can more easily cherry-pick extensions of interest from each
    update site.
  \item Expand the list of built-in ImageJ Ops with additional image processing
    and analysis routines, including Deep Learning approaches
    \cite{deep_learning} and novel algorithms from computer vision and
    statistics.
  \item Continue building bridges between ImageJ and other image processing
    frameworks such as \acrshort{opencv} \cite{opencv} and scikit-image
    \cite{scikit_image}.
  \item Integrate support for cloud computing frameworks such as Apache Spark
    \cite{apache_spark} running on platforms such as Amazon Web Services
    \cite{aws}.
  \item Continue supporting community requests for bug fixes, new features and
    image analysis advice.
  \item Continue migrating ImageJ resources into the main ImageJ wiki website,
    including the ImageJ User Guide \cite{imagej_user_guide}, ImageJ 1.x
    documentation \cite{imagej1_docs} and \acrshort{list}'s ImageJ Information
    and Documentation Portal \cite{imagej_docu}.
  \item Redesign ImageJ's bug submission system such that users can
    automatically submit an issue report to the correct location online
    whenever something goes wrong in the software.
\item Continue listening to, and working with, the user and developer
  community.
\end{itemize}

A detailed breakdown and discussion of each specific issue can be found on
GitHub, searchable from the unified ImageJ Search portal \cite{imagej_search}.

%% This should state clearly the main conclusions and provide an explanation of
%% the importance and relevance of the case, data, opinion, database or
%% software reported.

\section*{Conclusions}
Based on feedback from the existing ImageJ community, we have over the last
several years been designing and implementing ImageJ2, a radically improved
application that employs best practices and proven software engineering
approaches. ImageJ2 directly addresses two major needs, supporting applications
where: 1) the underlying ImageJ data engine was not sufficient to analyze
modern datasets; and 2) the lack of an underlying robust software design
impeded the addition of new functionality. This overhaul of ImageJ transforms
it into not only a powerful and flexible image processing and analysis
application in its own right, but also a framework for interoperability between
a plethora of external image visualization and analysis programs. ImageJ2
strengthens ImageJ as an ideal platform for scientific image analysis by: 1)
generalizing the ImageJ data model; 2) introducing a robust architecture
instrumental in building bridges across a swath of other image processing
tools; 3) remaining open source and cross-platform with permissive licensing,
enabling continued widespread adoption and extension; 4) building on the huge
collection of existing ImageJ plugins while enabling the creation of new
plugins with more powerful features; and 5) leveraging a correspondingly large
and diverse community to stand on one another's shoulders in pursuit of
knowledge and the betterment of the human condition.

%% If abbreviations are used in the text they should be defined in the text at
%% first use, and a list of abbreviations should be provided.

\printglossary[title=List of abbreviations,type=\acronymtype,style=long]

%%%%%%%%%%%%%%%%%%%%%%%%%%%%%%%%%%%%%%%%%%%%%%
%%                                          %%
%% Backmatter begins here                   %%
%%                                          %%
%%%%%%%%%%%%%%%%%%%%%%%%%%%%%%%%%%%%%%%%%%%%%%

\begin{backmatter}

\section*{Declarations}

\subsection*{Ethics approval and consent to participate}
Not applicable.

\subsection*{Consent for publication}
Not applicable.

\subsection*{Availability of data and material}

\textbf{Project name:} ImageJ\\
\textbf{Project home page:} https://imagej.net/\\
\textbf{Archived version:} net.imagej:imagej:2.0.0-rc-55\\
\textbf{Operating system(s):} Platform independent\\
\textbf{Programming language:} Java\\
\textbf{Other requirements:} Java 8 or higher\\
\textbf{License:} Simplified (2-clause) BSD\\
\textbf{Any restrictions to use by non-academics:} None

All data generated or analyzed during this study are included in this
published article.

\subsection*{Competing interests}
  The authors declare that they have no competing interests.

\subsection*{Funding}
  ImageJ2 was funded from 2010 through 2012 by the \acrfull{nigms} through the
  American Recovery and Reinvestment Act of 2009 \acrshort{nih} Research and
  Research Infrastructure ``Grand Opportunities'' Grant, ``Hardening'' of
  Biomedical Informatics/Computing Software for Robustness and Dissemination
  (Ref: RC2 GM092519-01), as well as a Wellcome Trust Strategic Award (Ref:
  095931). \acrshort{scifio} was funded by the National Science Foundation,
  award number 1148362. ImageJ2 projects were also funded by internal funding
  from the \acrlong{loci}, and the Morgridge Institute for Research.

\subsection*{Authors' contributions}
  CTR acted as the technical lead of the ImageJ2 project and primary architect
  of ImageJ2's software architecture. JS migrated key portions of
  \acrshort{fiji} into ImageJ2, including the Launcher and Updater components,
  and advised and improved upon many architectural aspects of ImageJ2,
  particularly the legacy layer. MCH served as the lead \acrshort{scifio}
  developer and contributed to all layers of the ImageJ software stack. BED
  developed substantial portions of the ImageJ2 codebase, including much of the
  legacy layer for backwards compatibility, prototype versions of ImageJ Ops
  for numerical processing, and many command implementations. AEW contributed
  to ImageJ Ops and the ImageJ-\acrshort{omero} integration layer. ETA made
  extensive edits and improvements to the manuscript. Lastly, as the primary
  principal investigator of ImageJ2, KWE directed and advised on all aspects of
  the project, including development directions and priorities. All authors
  contributed to, read, and approved the final manuscript.

\subsection*{Acknowledgements}
  Many people have contributed to the development of ImageJ2 on both technical
  and leadership levels. In particular, the authors gratefully thank and
  acknowledge the efforts of (in alphabetical order): Ignacio Arganda-Carreras,
  Michael Berthold, Tim-Oliver Buchholz, Jean-Marie Burel, Albert Cardona, Anne
  Carpenter, Christian Dietz, Richard Domander, Jan Eglinger, Gabriel Einsdorf,
  Adam Fraser, Aivar Grislis, Ulrik Günther, Robert Haase, Jonathan Hale, Kyle
  Harrington, Grant Harris, Stefan Helfrich, Martin Horn, Florian Jug, Lee
  Kamentsky, Gabriel Landini, Rick Lentz, Melissa Linkert, Mark Longair, Kevin
  Mader, Hadrien Mary, Kota Miura, Birgit Möller, Cyril Mongis, Josh Moore,
  Alec Neevel, Brian Northan, Rudolf Oldenbourg, Aparna Pal, Tobias Pietzsch,
  Stefan Posch, Stephan Preibisch, Loïc Royer, Stephan Saalfeld, Benjamin
  Schmid, Daniel Seebacher, Jason Swedlow, Jean-Yves Tinevez, Pavel Tomancak,
  Jay Warrick, Leon Yang, Yili Zhao and Michael Zinsmaier. We also thank the
  entire ImageJ community, especially those who contributed patch submissions,
  use cases, feature requests, and bug reports. A special thanks to Wayne
  Rasband for his tireless work on, and continuing maintenance of, ImageJ 1.x
  for these many years. Finally, our deep thanks to the \acrshort{nih}, whose
  initial funding of ImageJ2 in 2009 was instrumental in launching the project,
  as well as to all funding agencies and organizations who have supported the
  project's continued development \cite{imagej_funding}.

%%%%%%%%%%%%%%%%%%%%%%%%%%%%%%%%%%%
%%                               %%
%% Endnotes                      %%
%%                               %%
%%%%%%%%%%%%%%%%%%%%%%%%%%%%%%%%%%%
\section*{Endnotes}
  None.

%%%%%%%%%%%%%%%%%%%%%%%%%%%%%%%%%%%%%%%%%%%%%%%%%%%%%%%%%%%%%
%%                  The Bibliography                       %%
%%                                                         %%
%%  Bmc_mathpys.bst  will be used to                       %%
%%  create a .BBL file for submission.                     %%
%%  After submission of the .TEX file,                     %%
%%  you will be prompted to submit your .BBL file.         %%
%%                                                         %%
%%                                                         %%
%%  Note that the displayed Bibliography will not          %%
%%  necessarily be rendered by Latex exactly as specified  %%
%%  in the online Instructions for Authors.                %%
%%                                                         %%
%%%%%%%%%%%%%%%%%%%%%%%%%%%%%%%%%%%%%%%%%%%%%%%%%%%%%%%%%%%%%

% if your bibliography is in bibtex format, use those commands:
\bibliographystyle{bmc-mathphys} % Style BST file (bmc-mathphys, vancouver, spbasic).
\bibliography{imagej2}      % Bibliography file (usually '*.bib' )
% for author-year bibliography (bmc-mathphys or spbasic)
% a) write to bib file (bmc-mathphys only)
% @settings{label, options="nameyear"}
% b) uncomment next line
%\nocite{label}

% or include bibliography directly:
% \begin{thebibliography}
% \bibitem{b1}
% \end{thebibliography}

%%%%%%%%%%%%%%%%%%%%%%%%%%%%%%%%%%%
%%                               %%
%% Figures                       %%
%%                               %%
%% NB: this is for captions and  %%
%% Titles. All graphics must be  %%
%% submitted separately and NOT  %%
%% included in the Tex document  %%
%%                               %%
%%%%%%%%%%%%%%%%%%%%%%%%%%%%%%%%%%%

%%
%% Do not use \listoffigures as most will included as separate files

\section*{Figures and illustrations}
  %% Figure 1
  \begin{figure}[h!]
    \caption{Examples of image processing algorithms available in ImageJ Ops.}
    \begin{flushleft}
      Panel A (top left): 3D wireframe mesh of ImageJ's Bat Cochlea Volume
      sample dataset \cite{bat_cochlea_volume}, computed by the
      \texttt{geom.marchingCubes} op, an implementation of the marching cubes
      algorithm \cite{marching_cubes}, visualized using MeshLab \cite{meshlab}.
      Credit to Kyle Harrington for the figure, Tim-Oliver Buchholz for
      authoring the op, and Art Keating for the dataset. Panel B (top right):
      Richardson-Lucy Total Variation deconvolution \cite{richardson_lucy} of
      the Stellaris FISH dataset \#1 \cite{stellaris_fish}, computed by the
      \texttt{deconvolve.richardsonLucyTV} op. Credit to Brian Northan for
      authoring the op and figure \cite{bnorthan_ops_decon}, and George
      McNamara for the dataset. Panel C (bottom): Grayscale morphology and
      neighborhood filter operations on ImageJ's Lena sample image, using a
      diamond-shaped structuring element with radius 3. Credit to Jean-Yves
      Tinevez, Jonathan Hale and Leon Yang for authoring the ops.
    \end{flushleft}
  \end{figure}

  %% Figure 2
  \begin{figure}[h!]
    \caption{ImageJ update sites provide additional functionality to ImageJ.}
    \begin{flushleft}
      The Morphological Segmentation plugin, part of the MorphoLibJ plugin
      collection \cite{morpholibj}, easily segments the rings of ImageJ's Tree
      Rings sample dataset (panel A). The MorphoLibJ plugins are installed into
      the Fiji distribution of ImageJ by enabling the IJPB-plugins update site
      (panel B). Credit to David Legland and Ignacio Arganda-Carreras for
      authoring the plugins.
    \end{flushleft}
  \end{figure}

  %% Figure 3
  \begin{figure}[h!]
    \caption{ImageJ 1.x ~case logic compared to a unified ImgLib2
    implementation.}
    \begin{flushleft}
      Panel A (left) shows~the ImageJ 1.x implementation of a rolling ball
      background subtraction method, part of the
      \texttt{ij.plugin.filter.BackgroundSubtracter} class. Panel B (right)
      shows an equivalent implementation using ImgLib2, without the need for
      extensive case logic.
    \end{flushleft}
  \end{figure}

  %% Figure 4
  \begin{figure}[h!]
    \caption{Side-by-side comparison of ImageJ2-based user interfaces
    and integrations.}
    \begin{flushleft}
      Panel A (top left): ImageJFX, a JavaFX-based user interface built on
      ImageJ2. Panel B (top right): ImageJ2's default user interface, the
      ImageJ Legacy \acrshort{ui}, which wraps ImageJ 1.x. Panel C (bottom
      left): Example \acrshort{knime} workflow utilizing ImageJ2 image
      processing nodes. Panel D (middle right): Swing \acrshort{ui} prototype,
      closely modeled after ImageJ 1.x so that it remains familiar to existing
      users, in various Java ``\acrlong{laf}'' modes. Panel E (bottom right): A
      proof-of-concept Apache Pivot user interface. The ImageJFX and ImageJ
      Legacy \acrshortpl{ui} display an XY slice of ImageJ's Confocal Series
      sample dataset (dataset courtesy of Joel Sheffield), which has been
      rotated, smoothed and colorized.
    \end{flushleft}
  \end{figure}

  %% Figure 5
  \begin{figure}[h!]
    \caption{Comparison of pure ImageJ 1.x command with one using SciJava
    declarative syntax.}
    \begin{flushleft}
      Panel A (left) shows an ImageJ 1.x implementation of a plugin that copies
      slice labels from one image to another, as chosen by the user. Panel B
      (right) shows~the same plugin written using the SciJava declarative
      command syntax. Changed lines are highlighted in blue, new lines in
      green. The actual operation (the \texttt{copyLabels} method) is
      identical, but the routine for selecting which images to process is no
      longer necessary.
    \end{flushleft}
  \end{figure}

  %% Figure 6
  \begin{figure}[h!]
    \caption{Benchmarks of a simple addition operation with ImageJ Ops and
    ImageJ 1.x.}
    \begin{flushleft}
      Time performance comparison of simple addition operations between raw
      Java array manipulation, various math.add~operations of ImageJ Ops, and
      ImageJ 1.x's \textit{Process $\triangleright$ Math $\triangleright$
      Add\ldots} command. Benchmarks were run for 20 rounds on randomly
      generated \texttt{uint8} noise images dimensioned $15,000 \times 15,000$,
      using the JUnit Benchmarks framework, on a MacBook Pro (Retina, 15-inch,
      Mid 2015) running \acrshort{macos} Sierra 10.12 with 2.5 GHz Intel Core
      i7 processor and 16 GB 1600 MHz DDR3 memory. The routines which produced
      these results can be found in the ImageJ Ops test code, in the
      \texttt{AddOpBenchmarkTest} class of the
      \texttt{net.imagej.ops.benchmark} package.
    \end{flushleft}
  \end{figure}

  %% Figure 7
  \begin{figure}[h!]
    \caption{A mixed-world ImageJ 1.x + ImageJ2 script.}
    \begin{flushleft}
      This example Python script (panel D) uses ImageJ Ops to preprocess a
      confocal image and perform an automatic thresholding (panel B). ImageJ
      1.x's Analyze Particles routine is then called to isolate (panel A) and
      measure (panel C) foreground objects. Script contributed by Brian
      Northan, True North Intelligent Algorithms LLC. This script is available
      within ImageJ as a sample from the Tutorials submenu of the Script
      Editor's Templates menu.
    \end{flushleft}
  \end{figure}

%% HACK: Without this, some figures end up floating in subsequent sections.
\FloatBarrier

%%%%%%%%%%%%%%%%%%%%%%%%%%%%%%%%%%%
%%                               %%
%% Tables                        %%
%%                               %%
%%%%%%%%%%%%%%%%%%%%%%%%%%%%%%%%%%%

%% Use of \listoftables is discouraged.
%%
\section*{Tables and captions}
  None.

\section*{Supplemental material}

\subsection*{Figures and illustrations}
  %% Use "Figure S.#" naming scheme for supplemental figures
  \renewcommand\thefigure{S.\arabic{figure}}
  \setcounter{figure}{0}

  %% Figure S.1
  \begin{figure}[h!]
    \caption{Module execution in different contexts.}
    \begin{flushleft}
      When running a parameterized script (panel B) from the ImageJ user
      interface (panel A), a pop-up dialog box (panel C) enables the user to
      enter the name and age values; when running the script headless from the
      command line (panel D), input values are passed as arguments and output
      values echoed to the standard output stream.
    \end{flushleft}
  \end{figure}

  %% Figure S.2
  \begin{figure}[h!]
    \caption{Comparison of time performance across ImageJ 1.x and ImgLib2 data
    structures.}
    \begin{flushleft}
      For ten iterations, we ran a ``cheap'' per-pixel operation and an
      ``expensive'' operation on a 25 Mpx image stored in the ImageJ 1.x
      container, various ImgLib2 containers, and raw byte arrays. Panel A
      (left) shows the time (ms) it took to complete a ``cheap'' operations
      versus the loop iteration for each container. Panel B (right) shows the
      same information but for the time (ms) it took to complete the expensive
      operation.
    \end{flushleft}
  \end{figure}

\subsection*{Tables and captions}
  %% Use "Table S.#" naming scheme for supplemental tables
  \renewcommand\thetable{S.\arabic{table}}
  \setcounter{table}{0}

  %% Table S.1
  \begin{table}[h!]
    \caption{Built-in SciJava input widgets}
    \begin{tabular}{| l | l |}
      \hline
      \textbf{Java data type}                                                                                                            & \textbf{Widget type}                   \\ \hline
      \texttt{boolean} $\vert$ \texttt{Boolean}                                                                                          & checkbox                               \\ \hline
      \texttt{byte} $\vert$ \texttt{short} $\vert$ \texttt{int}     $\vert$ \texttt{long} $\vert$ \texttt{float} $\vert$ \texttt{double} & numeric field                          \\ \hline
      \texttt{Byte} $\vert$ \texttt{Short} $\vert$ \texttt{Integer} $\vert$ \texttt{Long} $\vert$ \texttt{Float} $\vert$ \texttt{Double} & numeric field                          \\ \hline
      \texttt{BigInteger} $\vert$ \texttt{BigDecimal}                                                                                    & numeric field                          \\ \hline
      \texttt{char} $\vert$ \texttt{Character} $\vert$ \texttt{String}                                                                   & text field                             \\ \hline
      \texttt{Dataset} $\vert$ \texttt{ImagePlus}                                                                                        & (\textgreater=2 images) drop-down list \\ \hline
      \texttt{ColorRGB}                                                                                                                  & color chooser                          \\ \hline
      \texttt{Date}                                                                                                                      & date chooser                           \\ \hline
      \texttt{File}                                                                                                                      & file chooser                           \\ \hline
    \end{tabular}
  \end{table}

  %% Table S.2
  \begin{table}[h!]
    \caption{Kinds and arities of special ops}
    \renewcommand{\arraystretch}{1.5}
    \begin{tabular}{| l | l | c | l | p{1.4in} |}
      \hline
      \textbf{Kind}              & \textbf{Stipulations} & \textbf{Arity} & \textbf{Interface}         & \textbf{Methods}                    \\ \hline
      \multirow{3}{*}{computer}  & \multirow{3}{1.8in}{
                                   \begin{itemize}[leftmargin=*]
                                     \renewcommand{\labelitemi}{-}
                                     \item Mutating the inputs is forbidden.\\
                                     \item The output and input references must differ (i.e., computers do not work in-place).\\
                                     \item The output's initial contents must not affect the value of the result.
                                   \end{itemize}
                                   }                     & 0              & \texttt{NullaryComputerOp} & \texttt{void compute(O)}            \\ \cline{3-5}
                                 &                       & 1              & \texttt{UnaryComputerOp}   & \texttt{void compute(I, O)}         \\ \cline{3-5}
                                 &                       & 2              & \texttt{BinaryComputerOp}  & \texttt{void compute(I1, I2, O)}    \\[0.45in] \hline
      \multirow{3}{*}{function}  & \multirow{3}{1.8in}{
                                   Mutating the inputs is forbidden.
                                   }                     & 0              & \texttt{NullaryFunctionOp} & \texttt{O calculate()}              \\ \cline{3-5}
                                 &                       & 1              & \texttt{UnaryFunctionOp}   & \texttt{O calculate(I)}             \\ \cline{3-5}
                                 &                       & 2              & \texttt{BinaryFunctionOp}  & \texttt{O calculate(I1, I2)}        \\ \hline
      \multirow{3}{*}{inplace}   & \multirow{3}{1.8in}{
                                   -
                                   }                     & 1              & \texttt{UnaryInplaceOp}    & \texttt{void mutate(O)}             \\ \cline{3-5}
                                 &                       & 1              & \texttt{BinaryInplace1Op}  & \texttt{void mutate1(O, I2)}        \\ \cline{3-5}
                                 &                       & 2              & \texttt{BinaryInplaceOp}   & \parbox[t]{2in}{
                                                                                                         \texttt{void mutate1(O, I2)}\\
                                                                                                         \texttt{void mutate2(I1, O)}
                                                                                                         }                                   \\[0.15in] \hline
      \multirow{3}{*}{hybrid CF} & \multirow{3}{1.8in}{
                                   Same as \textit{computer} and \textit{function} respectively.
                                   }                     & 0              & \texttt{NullaryHybridCF}   & \parbox[t]{2in}{
                                                                                                         \texttt{void compute(O)}\\
                                                                                                         \texttt{O calculate()}
                                                                                                         }                                   \\[0.15in] \cline{3-5}
                                 &                       & 1              & \texttt{UnaryHybridCF}     & \parbox[t]{2in}{
                                                                                                         \texttt{void compute(I, O)}\\
                                                                                                         \texttt{O calculate(I)}
                                                                                                         }                                   \\[0.15in] \cline{3-5}
                                 &                       & 2              & \texttt{BinaryHybridCF}    & \parbox[t]{2in}{
                                                                                                         \texttt{void compute(I1, I2, O)}\\
                                                                                                         \texttt{O calculate(I1, I2)}
                                                                                                         }                                   \\[0.15in] \hline
      \multirow{3}{*}{hybrid CI} & \multirow{3}{1.8in}{
                                   Same as \textit{computer} and \textit{inplace} respectively.
                                   }                     & 1              & \texttt{UnaryHybridCI}     & \parbox[t]{2in}{
                                                                                                         \texttt{void compute(I, O)}\\
                                                                                                         \texttt{void mutate(O)}
                                                                                                         }                                   \\[0.15in] \cline{3-5}
                                 &                       & 2              & \texttt{BinaryHybridCI1}   & \parbox[t]{2in}{
                                                                                                         \texttt{void compute(I1, I2, O)}\\
                                                                                                         \texttt{void mutate1(O, I2)}
                                                                                                         }                                   \\[0.15in] \cline{3-5}
                                 &                       & 2              & \texttt{BinaryHybridCI}    & \parbox[t]{2in}{
                                                                                                         \texttt{void compute(I1, I2, O)}\\
                                                                                                         \texttt{void mutate1(O, I2)}\\
                                                                                                         \texttt{void mutate2(I1, O)}
                                                                                                         }                                   \\[0.25in] \hline
      \multirow{3}{*}{hybrid CFI}& \multirow{3}{1.8in}{
                                   Same as \textit{computer}, \textit{function} and \textit{inplace} respectively.
                                   }                     & 1              & \texttt{UnaryHybridCFI}    & \parbox[t]{2in}{
                                                                                                         \texttt{void compute(I, O)}\\
                                                                                                         \texttt{O calculate(I)}\\
                                                                                                         \texttt{void mutate(O)}
                                                                                                         }                                   \\[0.25in] \cline{3-5}
                                 &                       & 2              & \texttt{BinaryHybridCFI1}  & \parbox[t]{2in}{
                                                                                                         \texttt{void compute(I1, I2, O)}\\
                                                                                                         \texttt{O calculate(I1, I2)}\\
                                                                                                         \texttt{void mutate1(O, I2)}
                                                                                                         }                                   \\[0.25in] \cline{3-5}
                                 &                       & 2              & \texttt{BinaryHybridCFI}   & \parbox[t]{2in}{
                                                                                                         \texttt{void compute(I1, I2, O)}\\
                                                                                                         \texttt{O calculate(O, I1, I2)}\\
                                                                                                         \texttt{void mutate1(O, I2)}\\
                                                                                                         \texttt{void mutate2(I1, O)}
                                                                                                         }                                   \\[0.4in] \hline
    \end{tabular}
  \end{table}

  %% Table S.3
  \begin{table}[h!]
    \caption{Image types supported by ImageJ}
    \begin{tabular}{| l | l | l | p{0.4in} | p{0.8in} | p{0.8in} | l | l |}
      \hline
      \textbf{Name}        & \textbf{Bit depth} & \textbf{Signedness} & \textbf{Values} & \textbf{Min. Value}     & \textbf{Max. Value}    & \textbf{ImageJ 1.x} & \textbf{ImageJ2} \\ \hline
      \texttt{bool}        & 1-bit              & N/A                 & boolean         & $false$                 & $true$                 & no                  & yes              \\ \hline
      \texttt{bit}         & 1-bit              & unsigned            & binary          & $0$                     & $1$                    & no                  & yes              \\ \hline
      \texttt{uint2}       & 2-bit              & unsigned            & integer         & $0$                     & $3$                    & no                  & yes              \\ \hline
      \texttt{uint4}       & 4-bit              & unsigned            & integer         & $0$                     & $15$                   & no                  & yes              \\ \hline
      \texttt{uint8}       & 8-bit              & unsigned            & integer         & $0$                     & $255$                  & yes                 & yes              \\ \hline
      \texttt{uint12}      & 12-bit             & unsigned            & integer         & $0$                     & $4,095$                & no                  & yes              \\ \hline
      \texttt{uint16}      & 16-bit             & unsigned            & integer         & $0$                     & $65,535$               & yes                 & yes              \\ \hline
      \texttt{uint32}      & 32-bit             & unsigned            & integer         & $0$                     & $2^{32} - 1$           & no                  & yes              \\ \hline
      \texttt{uint64}      & 64-bit             & unsigned            & integer         & $0$                     & $2^{64} - 1$           & no                  & yes              \\ \hline
      \texttt{uint128}     & 128-bit            & unsigned            & integer         & $0$                     & $2^{128} - 1$          & no                  & yes              \\ \hline
      \texttt{int8}        & 8-bit              & signed              & integer         & $-128$                  & $127$                  & partial*            & yes              \\ \hline
      \texttt{int16}       & 16-bit             & signed              & integer         & $-32,768$               & $32,767$               & partial*            & yes              \\ \hline
      \texttt{int32}       & 32-bit             & signed              & integer         & $-2^{31}$               & $2^{31} - 1$           & no                  & yes              \\ \hline
      \texttt{float32}     & 32-bit             & signed              & floating point  & \parbox[t]{0.8in}{
                                                                                          approx.\\
                                                                                          $-3.4 \times 10^{38}$
                                                                                          }                       & \parbox[t]{0.8in}{
                                                                                                                    approx.\\
                                                                                                                    $3.4 \times 10^{38}$
                                                                                                                    }                      & yes                 & yes              \\ \hline
      \texttt{float64}     & 64-bit             & signed              & floating point  & \parbox[t]{0.8in}{
                                                                                          approx.\\
                                                                                          $-1.8 \times 10^{308}$
                                                                                          }                       & \parbox[t]{0.8in}{
                                                                                                                    approx.\\
                                                                                                                    $1.8 \times 10^{308}$
                                                                                                                    }                      & no                  & yes              \\ \hline
      \texttt{cfloat32}    & 2 $\times$ 32-bit  & signed              & floating point  & \parbox[t]{0.8in}{
                                                                                          approx.\\
                                                                                          $(-3.4 \times 10^{38},\\
                                                                                          -3.4 \times 10^{38})$
                                                                                          }                       & \parbox[t]{0.8in}{
                                                                                                                    approx.\\
                                                                                                                    $(3.4 \times 10^{38},\\
                                                                                                                    3.4 \times 10^{38})$
                                                                                                                    }                      & no                  & yes              \\ \hline
      \texttt{cfloat64}    & 2 $\times$ 64-bit  & signed              & floating point  & \parbox[t]{0.8in}{
                                                                                          approx.\\
                                                                                          $(-1.8 \times 10^{308},\\
                                                                                          -1.8 \times 10^{308})$
                                                                                          }                       & \parbox[t]{0.8in}{
                                                                                                                    approx.\\
                                                                                                                    $(1.8 \times 10^{308},\\
                                                                                                                    1.8 \times 10^{308})$
                                                                                                                    }                      & no                  & yes              \\ \hline
      \texttt{bigint}      & unlimited          & signed              & integer         & none                    & none                   & no                  & yes              \\ \hline
      \texttt{bigdec}      & arbitrary          & signed              & decimal         & none                    & none                   & no                  & yes              \\ \hline
      \acrshort{rgb}       & 3 $\times$ 8-bit   & unsigned            & integer         & $(0, 0, 0)$             & $(255, 255, 255)$      & yes                 & legacy$^\dagger$ \\ \hline
      8-bit color          & 8-bit              & indexed             & integer         & $(0, 0, 0)$             & $(255, 255, 255)$      & yes                 & legacy$^\dagger$ \\ \hline
      custom               & any                & any                 & any             & any                     & any                    & no                  & yes              \\ \hline
    \end{tabular}
    \begin{flushleft}
      * ImageJ 1.x partially supports \texttt{int8} and \texttt{int16} types
      via an ``image calibration'' feature.

      $^\dagger$ ImageJ2 supports these image types only for backwards
      compatibility with ImageJ 1.x, since composite color mode with one
      \acrshort{lut} per channel achieves the same results in a more flexible
      way.
    \end{flushleft}
  \end{table}

\end{backmatter}
\end{document}
