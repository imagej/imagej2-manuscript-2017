%% BioMed_Central_Tex_Template_v1.06

%%IMPORTANT: do not delete the first line of this template
%%It must be present to enable the BMC Submission system to
%%recognise this template!!

%%%%%%%%%%%%%%%%%%%%%%%%%%%%%%%%%%%%%%%%%
%%                                     %%
%%  LaTeX template for BioMed Central  %%
%%     journal article submissions     %%
%%                                     %%
%%          <8 June 2012>              %%
%%                                     %%
%%                                     %%
%%%%%%%%%%%%%%%%%%%%%%%%%%%%%%%%%%%%%%%%%


%%%%%%%%%%%%%%%%%%%%%%%%%%%%%%%%%%%%%%%%%%%%%%%%%%%%%%%%%%%%%%%%%%%%%
%%                                                                 %%
%% For instructions on how to fill out this Tex template           %%
%% document please refer to Readme.html and the instructions for   %%
%% authors page on the biomed central website                      %%
%% http://www.biomedcentral.com/info/authors/                      %%
%%                                                                 %%
%% Please do not use \input{...} to include other tex files.       %%
%% Submit your LaTeX manuscript as one .tex document.              %%
%%                                                                 %%
%% All additional figures and files should be attached             %%
%% separately and not embedded in the \TeX\ document itself.       %%
%%                                                                 %%
%% BioMed Central currently use the MikTex distribution of         %%
%% TeX for Windows) of TeX and LaTeX.  This is available from      %%
%% http://www.miktex.org                                           %%
%%                                                                 %%
%%%%%%%%%%%%%%%%%%%%%%%%%%%%%%%%%%%%%%%%%%%%%%%%%%%%%%%%%%%%%%%%%%%%%

%%% additional documentclass options:
%  [doublespacing]
%  [linenumbers]   - put the line numbers on margins

%%% loading packages, author definitions

%\documentclass[twocolumn]{bmcart}% uncomment this for twocolumn layout and comment line below
\documentclass{bmcart}

%%% Load packages
%\usepackage{amsthm,amsmath}
%\RequirePackage{natbib}
%\RequirePackage[authoryear]{natbib}% uncomment this for author-year bibliography
%\RequirePackage{hyperref}
\usepackage[utf8]{inputenc} %unicode support
%\usepackage[applemac]{inputenc} %applemac support if unicode package fails
%\usepackage[latin1]{inputenc} %UNIX support if unicode package fails

%% We use the glossaries package for the list of acronyms.
\usepackage[nonumberlist]{glossaries}
\renewcommand{\glossarypreamble}{\footnotesize}
\makeglossaries
\newacronym{alida}{Alida}{Advanced Library for Integrated Development of data analysis Applications}
\newacronym{api}{API}{Application Programming Interface}
\newacronym{awt}{AWT}{Abstract Windowing Toolkit}
\newacronym{bar}{BAR}{Broadly Applicable Routines}
\newacronym{bmc}{BMC}{BioMed Central}
\newacronym{bsd}{BSD}{Berkeley Software Distribution}
\newacronym{ci}{CI}{continuous integration}
\newacronym{cpu}{CPU}{Central Processing Unit}
\newacronym{di}{DI}{dependency injection}
\newacronym{dry}{DRY}{Don't Repeat Yourself}
\newacronym{epfl}{EPFL}{École Polytechnique Fédérale de Lausanne}
\newacronym{fiji}{Fiji}{Fiji Is Just ImageJ}
\newacronym{gimp}{GIMP}{GNU Image Manipulation Program}
\newacronym{gnu}{GNU}{GNU's Not Unix}
\newacronym{gpu}{GPU}{Graphics Processing Unit}
\newacronym{gtk}{GTK}{GIMP ToolKit}
\newacronym{http}{HTTP}{Hypertext Transfer Protocol}
\newacronym{io}{I/O}{Input/Output}
\newacronym{ide}{IDE}{Integrated Development Environment}
\newacronym{ioc}{IoC}{inversion of control}
\newacronym{itk}{ITK}{Insight ToolKit}
\newacronym{jar}{JAR}{Java ARchive}
\newacronym{jit}{JIT}{Just-In-Time compiler}
\newacronym{jvm}{JVM}{Java Virtual Machine}
\newacronym{knime}{KNIME}{KoNstanz Information MinEr}
\newacronym{knip}{KNIP}{KNIME Image Processing}
\newacronym{laf}{L\&F}{Look \& Feel}
\newacronym{list}{LIST}{Luxembourg Institute of Science and Technology}
\newacronym{lut}{LUT}{color lookup table}
\newacronym{loci}{LOCI}{Laboratory for Optical and Computational Instrumentation}
\newacronym{macos}{macOS}{Macintosh Operating System}
\newacronym{matlab}{MATLAB}{MATrix LABoratory}
\newacronym{mitobo}{MiToBo}{Microscopy image analysis ToolBox}
\newacronym{nih}{NIH}{National Institutes of Health}
\newacronym{ome}{OME}{Open Microscopy Environment}
\newacronym{omero}{OMERO}{OME Remote Objects}
\newacronym{opencv}{OpenCV}{Open source Computer Vision library}
\newacronym{pom}{POM}{Project Object Model}
\newacronym{ram}{RAM}{Random-Access Memory}
\newacronym{rest}{RESTful}{REpresentational State Transfer}
\newacronym[longplural={regions of interest}]{roi}{ROI}{region of interest}
\newacronym{scifio}{SCIFIO}{SCientific Image File Input and Output}
\newacronym{scp}{SCP}{Secure CoPy}
\newacronym{sftp}{SFTP}{Secure File Transfer Protocol}
\newacronym{ssh}{SSH}{Secure SHell}
\newacronym{swt}{SWT}{Standard Widget Toolkit}
\newacronym{ui}{UI}{user interface}
\newacronym{uri}{URI}{Uniform Resource Identifier}
\newacronym{url}{URL}{Uniform Resource Locator}
\newacronym{webdav}{WebDAV}{Web Distributed Authoring and Versioning}
\newacronym{xml}{XML}{eXtensible Markup Language}

%%%%%%%%%%%%%%%%%%%%%%%%%%%%%%%%%%%%%%%%%%%%%%%%%
%%                                             %%
%%  If you wish to display your graphics for   %%
%%  your own use using includegraphic or       %%
%%  includegraphics, then comment out the      %%
%%  following two lines of code.               %%
%%  NB: These line *must* be included when     %%
%%  submitting to BMC.                         %%
%%  All figure files must be submitted as      %%
%%  separate graphics through the BMC          %%
%%  submission process, not included in the    %%
%%  submitted article.                         %%
%%                                             %%
%%%%%%%%%%%%%%%%%%%%%%%%%%%%%%%%%%%%%%%%%%%%%%%%%


\def\includegraphic{}
\def\includegraphics{}



%%% Put your definitions there:
\startlocaldefs
\endlocaldefs


%%% Begin ...
\begin{document}

%%% Start of article front matter
\begin{frontmatter}

\begin{fmbox}
\dochead{Software}

%%%%%%%%%%%%%%%%%%%%%%%%%%%%%%%%%%%%%%%%%%%%%%
%%                                          %%
%% Enter the title of your article here     %%
%%                                          %%
%%%%%%%%%%%%%%%%%%%%%%%%%%%%%%%%%%%%%%%%%%%%%%

\title{ImageJ2: ImageJ for the next generation of scientific image data}

%%%%%%%%%%%%%%%%%%%%%%%%%%%%%%%%%%%%%%%%%%%%%%
%%                                          %%
%% Enter the authors here                   %%
%%                                          %%
%% Specify information, if available,       %%
%% in the form:                             %%
%%   <key>={<id1>,<id2>}                    %%
%%   <key>=                                 %%
%% Comment or delete the keys which are     %%
%% not used. Repeat \author command as much %%
%% as required.                             %%
%%                                          %%
%%%%%%%%%%%%%%%%%%%%%%%%%%%%%%%%%%%%%%%%%%%%%%

\author[
   addressref={aff1}
]{\inits{CTR}\fnm{Curtis T} \snm{Rueden}}
\author[
   addressref={aff1,aff2}
]{\inits{JS}\fnm{Johannes} \snm{Schindelin}}
\author[
   addressref={aff1}
]{\inits{MCH}\fnm{Mark C} \snm{Hiner}}
\author[
   addressref={aff1}
]{\inits{BED}\fnm{Barry E} \snm{DeZonia}}
\author[
   addressref={aff1}
]{\inits{AEW}\fnm{Alison E} \snm{Walter}}
\author[
   addressref={aff1,aff2},
   email={eliceiri@wisc.edu}
]{\inits{KWE}\fnm{Kevin W} \snm{Eliceiri}}

%%%%%%%%%%%%%%%%%%%%%%%%%%%%%%%%%%%%%%%%%%%%%%
%%                                          %%
%% Enter the authors' addresses here        %%
%%                                          %%
%% Repeat \address commands as much as      %%
%% required.                                %%
%%                                          %%
%%%%%%%%%%%%%%%%%%%%%%%%%%%%%%%%%%%%%%%%%%%%%%

\address[id=aff1]{%                           % unique id
  \orgname{Laboratory for Optical and Computational Instrumentation, University of Wisconsin at Madison},
  \city{Madison},
  \state{Wisconsin},
  \cny{USA}
}
\address[id=aff2]{%
  \orgname{Morgridge Institute for Research},
  \city{Madison},
  \state{Wisconsin},
  \cny{USA}
}

\end{fmbox}% comment this for two column layout

%%%%%%%%%%%%%%%%%%%%%%%%%%%%%%%%%%%%%%%%%%%%%%
%%                                          %%
%% The Abstract begins here                 %%
%%                                          %%
%% Please refer to the Instructions for     %%
%% authors on http://www.biomedcentral.com  %%
%% and include the section headings         %%
%% accordingly for your article type.       %%
%%                                          %%
%%%%%%%%%%%%%%%%%%%%%%%%%%%%%%%%%%%%%%%%%%%%%%

%% The Abstract of the manuscript should not exceed 350 words and must be
%% structured into separate sections: Background, the context and purpose of
%% the study; Results, the main findings; Conclusions, brief summary and
%% potential implications.
%%
%% Please do not use abbreviations or references in the abstract. Please
%% see also our guide for writing an easily accessible abstract.

\begin{abstractbox}

\begin{abstract} % abstract
\parttitle{Background}
  ImageJ is an image analysis program extensively used in the biological
  sciences and beyond. Due to its ease of use, recordable macro language, and
  extensible plug-in architecture, ImageJ enjoys contributions from
  non-programmers, amateur programmers, and professional developers alike---a
  feat that arguably no other image analysis package, commercial or open
  source, has yet achieved. Enabling such a diversity of contributors has
  resulted in a unique community that spans the biological and physical
  sciences, making ImageJ an invaluable resource across countless disciplines.
  However, a rapidly growing user base, diverging plugin suites, and technical
  limitations have revealed a clear need for a concerted software engineering
  effort to support emerging imaging paradigms, to ensure the software's
  ability to handle the requirements of modern science.

\parttitle{Results}
  Our goal was ambitious: to create a future-proof tool without sacrificing the
  existing community. We rebuilt ImageJ from the ground up, engineering a
  powerful plugin mechanism that facilitates extensibility at every level. This
  next-generation ImageJ, called ``ImageJ2'' in places where the distinction
  matters, provides a host of new functionality. It separates concerns, fully
  decoupling the data model from the user interface. It emphasizes integration
  with external applications to maximize interoperability. Its robust new
  plugin framework allows everything from image formats, to scripting
  languages, to visualization to be extended by the community. The redesigned
  data model supports arbitrarily large, N-dimensional datasets, which are
  increasingly common in modern image acquisition. Despite the scope of these
  changes, backwards compatibility is maintained such that this new
  functionality can be seamlessly integrated with the classic ImageJ interface,
  allowing users and developers to migrate to these new methods at their own
  pace.

\parttitle{Conclusions}
  Scientific imaging benefits from open-source programs that advance new method
  development and deployment to a diverse audience. ImageJ has excelled in this
  regard; however, due to new and emerging challenges, it is at a critical
  development crossroads. The described improvements provide a framework
  adaptable to future needs, enabling continued success and innovation. Future
  efforts will focus on implementing new algorithms in this framework and
  expanding collaborations with other popular scientific software suites.
\end{abstract}

%%%%%%%%%%%%%%%%%%%%%%%%%%%%%%%%%%%%%%%%%%%%%%
%%                                          %%
%% The keywords begin here                  %%
%%                                          %%
%% Put each keyword in separate \kwd{}.     %%
%%                                          %%
%%%%%%%%%%%%%%%%%%%%%%%%%%%%%%%%%%%%%%%%%%%%%%

\begin{keyword}
\kwd{ImageJ}
\kwd{ImageJ2}
\kwd{image processing}
\kwd{N-dimensional}
\kwd{interoperability}
\kwd{extensibility}
\kwd{reproducibility}
\kwd{open source}
\kwd{open development}
\end{keyword}

% MSC classifications codes, if any
%\begin{keyword}[class=AMS]
%\kwd[Primary ]{}
%\kwd{}
%\kwd[; secondary ]{}
%\end{keyword}

\end{abstractbox}
%
%\end{fmbox}% uncomment this for twcolumn layout

\end{frontmatter}

%%%%%%%%%%%%%%%%%%%%%%%%%%%%%%%%%%%%%%%%%%%%%%
%%                                          %%
%% The Main Body begins here                %%
%%                                          %%
%% Please refer to the instructions for     %%
%% authors on:                              %%
%% http://www.biomedcentral.com/info/authors%%
%% and include the section headings         %%
%% accordingly for your article type.       %%
%%                                          %%
%% See the Results and Discussion section   %%
%% for details on how to create sub-sections%%
%%                                          %%
%% use \cite{...} to cite references        %%
%%  \cite{koon} and                         %%
%%  \cite{oreg,khar,zvai,xjon,schn,pond}    %%
%%  \nocite{smith,marg,hunn,advi,koha,mouse}%%
%%                                          %%
%%%%%%%%%%%%%%%%%%%%%%%%%%%%%%%%%%%%%%%%%%%%%%

%%%%%%%%%%%%%%%%%%%%%%%%% start of article main body
% <put your article body there>

%% The Background section should be written in a way that is accessible to
%% researchers without specialist knowledge in that area and must clearly
%% state - and, if helpful, illustrate - the background to the research and its
%% aims. It should clearly described the relevant context and the specific
%% issue which the software described is intended to address.

\section*{Background}
ImageJ \cite{imagej_history} is a powerful, oft-referenced platform for image
processing, developed by Wayne Rasband at the \acrfull{nih}. Since its initial
release in 1997, ImageJ has proven paramount in many scientific endeavors and
projects, particularly those within the life sciences \cite{imagej_review}.
Over the past nineteen years, the program has evolved far beyond its originally
intended scope. After such an extended period of sustained growth, any software
project benefits from a subsequent period of scrutiny and refactoring; ImageJ
is no exception. Such restructuring helps the program to remain accessible to
newcomers, powerful enough for experts, and relevant to the demands of its
ever-growing community. As such, we have developed ImageJ2: a total redesign of
the previous incarnation (hereafter ``ImageJ 1.x''), which builds on the
original's successful qualities while improving its core architecture to
encompass the scientific demands of the decades to come. Key motivations for
the development of ImageJ2 include:

\begin{enumerate}
  \item \textbf{Supporting the next generation of image data.} Over time, the
    infrastructure of image acquisition has grown in sophistication and
    complexity. For example, in the field of microscopy we were once limited to
    single image planes. However, with modern techniques we can record much
    more information: physical location in time and space (X, Y, Z, time),
    lifetime histograms across a range of spectral emission channels,
    polarization state of light, phase and frequency, angles of rotation (e.g.,
    in light sheet fluorescence microscopy), and high-throughput screens, just
    to name a few. The ImageJ infrastructure needed improvement to work
    effectively with these new modes of image data.

  \item \textbf{Enabling new software collaborations.} The field of software
    engineering has seen an explosion of available development tools and
    infrastructure, and it is no longer realistic to expect a single standalone
    application to remain universally relevant. ImageJ2 takes collaboration to
    new levels, via a modular framework enabling effortless integration with
    external software suites. For example, ImageJ2 plugins can be automatically
    translated to nodes in a \acrshort{knime} \cite{knime} workflow, and new
    JavaFX \cite{javafx} web interfaces can be built on top of ImageJ. Once
    such bridges are established, algorithms can be written once and reused in
    a variety of contexts without any extra work.

  \item \textbf{Broadening the ImageJ community.} Though initially developed
    for the life sciences, ImageJ has the potential to be a powerful tool for
    any field that benefits from image visualization, processing, and analysis:
    earth sciences, astronomy, fluid dynamics, computer vision, signal
    processing, etc. We enhance ImageJ's impact throughout the scientific
    community by providing central online resources which are comprehensive,
    consistently structured and easily editable by the community.
\end{enumerate}

From these motivations emerge the six pillars of the ImageJ2 mission
statement:

\begin{itemize}
  \item \textbf{Design} the next generation of ImageJ, driven by the needs of
    the community.
  \item \textbf{Collaborate} across organizations, fostering open development
    through sharing and reuse.
  \item \textbf{Broaden} ImageJ's usefulness and relevance across many
    disciplines of the scientific community.
  \item \textbf{Maintain} backwards compatibility with existing ImageJ
    functionality.
  \item \textbf{Unify} online resources to a central location for the ImageJ
    community.
  \item \textbf{Lead} ImageJ development with a clear vision.
\end{itemize}

%% The Design Goals section discusses ImageJ2's _goals_ only, not the
%% _reality_ of what it does. So it uses language like "must" and "should"
%% and "strives to" rather than "does" and "is". The latter (especially
%% details thereof) is for Implementation and Results below only.

\subsection*{Design Goals}
The central technical design goals of ImageJ2 can be divided into seven key
categories: functionality, extensibility, reproducibility, usability,
performance, compatibility and community. In this section, we discuss the goals
of ImageJ2 from its outset; for how these goals have been met in practice, see
the subsequent sections.

\subsubsection*{Functionality}
The overriding principle of ImageJ2 is to create \textbf{\textit{powerful}}
software, capable of meeting the expanding requirements of an ever-more-complex
landscape of scientific image processing and analysis for the foreseeable
future. As such, ImageJ needs to be more than just an application: it must be a
\textbf{\textit{modular}}, multi-layered set of functions with each layer
encapsulated and building upon lower layers. In computer science terminology,
ImageJ2 strives to have a proper \textbf{\textit{separation of concerns}}
between data model and display thereof, enabling use within a wide variety of
scenarios, such as headless operation---i.e., running remotely on a server,
cluster or cloud without a graphical \acrfull{ui}.

At its core, ImageJ2 aims to provide robust support for
\textbf{\textit{N-dimensional}} image data, to support domains with dimensions
beyond time and space. Examples include: multispectral and hyperspectral
images, fluorescence lifetime measured in the time or frequency domains,
multi-angle data from acquisition modalities such as light sheet fluorescence
microscopy, multi-position data from High Content Screens, and experiments
using polarized light. In general, the design must be robust enough to express
any newly emerging modalities within its infrastructure.

Finally, it is not sufficient to provide a modular framework---ImageJ2 must
also provide \textbf{\textit{built-in routines}} as default behavior for
standard tasks in image processing and analysis. These core plugins must span a
wealth of algorithms for image processing and analysis, image visualization,
and image file import and export. Such built-in features ensure users have an
application they can apply out-of-the-box.

\subsubsection*{Extensibility}
The quality that makes ImageJ most powerful---its greatest strength---is its
\textbf{\textit{extensibility}}. From its inception \cite{imagej_history},
ImageJ 1.x has had a mechanism by which users can develop their own plugins and
macros to extend its capabilities. Two decades later, a plethora of such
plugins and macros have been shared and published \cite{imagej_ecosystem}. It
is paramount that ImageJ2 maintains this ease of modification and extension by
its user community, and furthermore leverages its improved separation of
concerns to actually make user extension easier and more powerful; e.g., if
image processing plugins are agnostic to user interface, new interfaces can be
developed without a loss of functionality.

A related preeminent concern is \textbf{\textit{interoperability}}. There is no
silver bullet in image processing. No matter how powerful ImageJ becomes or how
many extensions exist, there will always be powerful and useful alternative
tools available. Users benefit most when information can easily be exchanged
between such tools. One of ImageJ2's primary motivations is to enable usage of
ImageJ code from within other applications and to support open standards for
data storage and exchange.

\subsubsection*{Reproducibility}
For ImageJ to be truly useful to the scientific community, it must be not only
technically feasible to extend, but also socially feasible, without legal
obstacles or other restrictions preventing the free exchange of scientific
ideas. To that end, ImageJ must be not only open source, but offer full
\textbf{\textit{reproducibility}}, following an \textbf{\textit{open
development process}} which we believe is an optimal fit for open scientific
inquiry \cite{software_usability}. We want to enable the community to not just
use ImageJ, but also to build upon it, with all project resources---revision
history, project roadmap, community contribution process, etc.---publicly
accessible, and development discussions taking place in public, archived
communication channels so that interested parties can remain informed of and
contribute to the project's future directions. Such transparency also
facilitates sensible, defensible software development processes and fosters
responsibility amongst those involved in the ImageJ project.

\subsubsection*{Usability}
Modular systems composed of many components often have a corresponding increase
in conceptual complexity, making them harder to understand and use. To avoid
this pitfall, ImageJ2 employs the idea of complexity minimization: seeking
\textbf{\textit{sensible defaults}} that make simple things easy, but difficult
things still possible. The lowest-level software layers should define the
program's full power, while each subsequent layer reduces visible complexity by
choosing default parameters suitable for common tasks. The highest levels
should provide users with the simplicity of a ``big green button,'' performing
the most commonly desired tasks with ease---the powerful inner machinery
remaining unseen, yet accessible when needed.

To bridge the gap between extensibility and usability, there must be a painless
process of installing new functionality: a built-in, configurable
\textbf{\textit{automatic update mechanism}} to manage extensions and keep the
software up-to-date. This update mechanism must be scalable and distributed,
such that software developers can publish their own extensions on their own
websites, without needing to obtain permission from a central authority.

\subsubsection*{Performance}
N-dimensional images and the ever-expanding size of datasets increase the
computation requirements placed on analysis routines. For ImageJ2 to succeed,
it must accomplish its goals without negatively impacting performance
\textbf{\textit{efficiency}} in time---e.g., \acrfull{cpu} and
\acrfull{gpu}---or space---e.g., \acrfull{ram} and disk. Furthermore, to ensure
ImageJ2 meets performance needs for a wide variety of use cases, it should
offer choices surrounding usage of available resources, as well as sensible
defaults for balancing performance in common scenarios.

Another key consideration for performance is \textbf{\textit{scalability}}:
ImageJ must be capable of operating on increasingly huge datasets. In cloud
computing, this requirement is often met via elasticity: the ability to
transparently provision additional computing resources---i.e., throw more
computers at the problem \cite{hardware_is_cheap}. We are at the dawn of the
``Big Data'' era of computing, where both computation and storage are scalable
resources which can be purchased from remote server farms. Software like ImageJ
which hopes to remain effective for serious scientific inquiry into the coming
decades must be architected so that its algorithms scale well to increasingly
large data processed in parallel across increasingly large numbers of
\acrshort{cpu} and \acrshort{gpu} cores.

\subsubsection*{Compatibility}
There are a vast number of existing extensions---plugins, macros, and
scripts---for the original ImageJ 1.x application which have proven extremely
useful to the user community \cite{imagej_ecosystem}. ImageJ2 must continue to
support these extensions as faithfully as possible, while also providing a
clear incremental migration path to take advantage of the new framework.

\subsubsection*{Community}
The principal non-technical goal of ImageJ2 is to serve the ImageJ community as
it evolves and grows; to that end, several community-oriented technical goals
naturally follow. The ImageJ project must provide \textbf{\textit{unified
online resources}} including a central community-editable website, discussion
forum, and online technical resources for managing community extensions of
ImageJ. And the ImageJ application itself must work in concert with these
resources---e.g., users should be able to report bugs directly to online issue
tracking systems when something goes wrong.

%% This should include a description of the overall architecture of the
%% software implementation, along with details of any critical issues and
%% how they were addressed.

\section*{Implementation}
Text for this section.

%% The user interface should be described and a discussion of the intended uses
%% of the software, and the benefits that are envisioned, should be included,
%% together with data on how its performance and functionality compare with,
%% and improve, on functionally similar existing software. A case study of the
%% use of the software may be presented. The planned future development of new
%% features, if any, should be mentioned.

\section*{Results and Discussion}
Text for this section.

%% This should state clearly the main conclusions and provide an explanation of
%% the importance and relevance of the case, data, opinion, database or
%% software reported.

\section*{Conclusions}
Text for this section.

%% If abbreviations are used in the text they should be defined in the text at
%% first use, and a list of abbreviations should be provided.

\printglossary[title=List of abbreviations,type=\acronymtype,style=long]

%%%%%%%%%%%%%%%%%%%%%%%%%%%%%%%%%%%%%%%%%%%%%%
%%                                          %%
%% Backmatter begins here                   %%
%%                                          %%
%%%%%%%%%%%%%%%%%%%%%%%%%%%%%%%%%%%%%%%%%%%%%%

\begin{backmatter}

\section*{Declarations}

\subsection*{Competing interests}
  The authors declare that they have no competing interests.

\subsection*{Author's contributions}
  CTR acted as the technical lead of the ImageJ2 project and primary architect
  of ImageJ2's software architecture. JS migrated key portions of
  \acrshort{fiji} into ImageJ2, including the Launcher and Updater components,
  and advised and improved upon many architectural aspects of ImageJ2,
  particularly the legacy layer. MCH served as the lead \acrshort{scifio}
  developer and contributed to all layers of the ImageJ software stack. BED
  developed substantial portions of the ImageJ2 codebase, including much of the
  legacy layer for backwards compatibility, prototype versions of ImageJ Ops
  for numerical processing, and many command implementations. AEW contributed
  to ImageJ Ops and the ImageJ-\acrshort{omero} integration layer. Lastly, as
  the primary principal investigator of ImageJ2, KWE directed and advised on
  all aspects of the project, including development directions and priorities.
  All authors contributed to, read, and approved the final manuscript.

\subsection*{Acknowledgements}
  Many people have contributed to the development of ImageJ2 on both technical
  and leadership levels. In particular, the authors gratefully thank and
  acknowledge the efforts of (in alphabetical order): Ellen T. Arena, Ignacio
  Arganda-Carreras, Michael Berthold, Tim-Oliver Buchholz, Jean-Marie Burel,
  Albert Cardona, Anne Carpenter, Christian Dietz, Richard Domander, Jan
  Eglinger, Gabriel Einsdorf, Adam Fraser, Aivar Grislis, Ulrik Günther, Robert
  Haase, Jonathan Hale, Kyle Harrington, Grant Harris, Stefan Helfrich, Martin
  Horn, Florian Jug, Lee Kamentsky, Gabriel Landini, Rick Lentz, Melissa
  Linkert, Mark Longair, Kevin Mader, Hadrien Mary, Kota Miura, Birgit Möller,
  Cyril Mongis, Josh Moore, Alec Neevel, Brian Northan, Rudolf Oldenbourg,
  Aparna Pal, Tobias Pietzsch, Stefan Posch, Stephan Preibisch, Loïc Royer,
  Stephan Saalfeld, Benjamin Schmid, Daniel Seebacher, Jason Swedlow, Jean-Yves
  Tinevez, Pavel Tomancak, Jay Warrick, Leon Yang, Yili Zhao and Michael
  Zinsmaier. We also thank the entire ImageJ community, especially those who
  contributed patch submissions, use cases, feature requests, and bug reports.
  A special thanks to Wayne Rasband for his tireless work on, and continuing
  maintenance of, ImageJ 1.x for these many years. Finally, our deep thanks to
  the \acrshort{nih}, whose initial funding of ImageJ2 in 2009 was instrumental
  in launching the project, as well as to all funding agencies and
  organizations who have supported the project's continued development.
  \cite{imagej_funding}

%%%%%%%%%%%%%%%%%%%%%%%%%%%%%%%%%%%
%%                               %%
%% Endnotes                      %%
%%                               %%
%%%%%%%%%%%%%%%%%%%%%%%%%%%%%%%%%%%
\section*{Endnotes}
  None.

%%%%%%%%%%%%%%%%%%%%%%%%%%%%%%%%%%%%%%%%%%%%%%%%%%%%%%%%%%%%%
%%                  The Bibliography                       %%
%%                                                         %%
%%  Bmc_mathpys.bst  will be used to                       %%
%%  create a .BBL file for submission.                     %%
%%  After submission of the .TEX file,                     %%
%%  you will be prompted to submit your .BBL file.         %%
%%                                                         %%
%%                                                         %%
%%  Note that the displayed Bibliography will not          %%
%%  necessarily be rendered by Latex exactly as specified  %%
%%  in the online Instructions for Authors.                %%
%%                                                         %%
%%%%%%%%%%%%%%%%%%%%%%%%%%%%%%%%%%%%%%%%%%%%%%%%%%%%%%%%%%%%%

% if your bibliography is in bibtex format, use those commands:
\bibliographystyle{bmc-mathphys} % Style BST file (bmc-mathphys, vancouver, spbasic).
\bibliography{imagej2}      % Bibliography file (usually '*.bib' )
% for author-year bibliography (bmc-mathphys or spbasic)
% a) write to bib file (bmc-mathphys only)
% @settings{label, options="nameyear"}
% b) uncomment next line
%\nocite{label}

% or include bibliography directly:
% \begin{thebibliography}
% \bibitem{b1}
% \end{thebibliography}

%%%%%%%%%%%%%%%%%%%%%%%%%%%%%%%%%%%
%%                               %%
%% Figures                       %%
%%                               %%
%% NB: this is for captions and  %%
%% Titles. All graphics must be  %%
%% submitted separately and NOT  %%
%% included in the Tex document  %%
%%                               %%
%%%%%%%%%%%%%%%%%%%%%%%%%%%%%%%%%%%

%%
%% Do not use \listoffigures as most will included as separate files

\section*{Figures}
  \begin{figure}[h!]
  \caption{\csentence{Sample figure title.}
      A short description of the figure content
      should go here.}
      \end{figure}

\begin{figure}[h!]
  \caption{\csentence{Sample figure title.}
      Figure legend text.}
      \end{figure}

%%%%%%%%%%%%%%%%%%%%%%%%%%%%%%%%%%%
%%                               %%
%% Tables                        %%
%%                               %%
%%%%%%%%%%%%%%%%%%%%%%%%%%%%%%%%%%%

%% Use of \listoftables is discouraged.
%%
\section*{Tables}
\begin{table}[h!]
\caption{Sample table title. This is where the description of the table should go.}
      \begin{tabular}{cccc}
        \hline
           & B1  &B2   & B3\\ \hline
        A1 & 0.1 & 0.2 & 0.3\\
        A2 & ... & ..  & .\\
        A3 & ..  & .   & .\\ \hline
      \end{tabular}
\end{table}

%%%%%%%%%%%%%%%%%%%%%%%%%%%%%%%%%%%
%%                               %%
%% Additional Files              %%
%%                               %%
%%%%%%%%%%%%%%%%%%%%%%%%%%%%%%%%%%%

\section*{Additional Files}
  \subsection*{Additional file 1 --- Sample additional file title}
    Additional file descriptions text (including details of how to
    view the file, if it is in a non-standard format or the file extension).  This might
    refer to a multi-page table or a figure.

  \subsection*{Additional file 2 --- Sample additional file title}
    Additional file descriptions text.


\end{backmatter}
\end{document}
